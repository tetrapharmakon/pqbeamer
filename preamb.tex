\usepackage[utf8]{inputenc}
\usepackage[T1]{fontenc}
\usepackage{graphicx}
\usepackage{grffile}
\usepackage{longtable}
\usepackage{wrapfig}
\usepackage{rotating}
\usepackage[normalem]{ulem}
\usepackage{amsmath}
\usepackage{textcomp}
\usepackage{amssymb}
\usepackage{capt-of}
\usepackage{hyperref}

\newcommand{\xlongertwoheadrightarrow}[1]{%
    \mathrel{%
        \mkern-15.25mu
        \begin{tikzcd}[row sep=3.6em, column sep=1.8em, ampersand replacement=\&]
            {}
            \arrow[r, "#1", two heads] \&
            {}
        \end{tikzcd}
        \mkern-16.25mu
    }
}
\let\xlongertwoheadsrightarrow\xlongertwoheadrightarrow
% \let\twoheadrightarrow\xlongertwoheadrightarrow

\newcommand{\wlim}[1]{\mathrm{lim}^{#1}}
\newcommand{\wcolim}[1]{\mathrm{colim}^{#1}}
\newcommand{\procirc}{\mathbin{\diamond}}
\newcommand{\ph}{\mathsf{h}}
\newcommand{\WeightedEnd}[2]{\int_{#1}^{#2}}
\newcommand{\WeightedCoend}[2]{\int^{#1}_{#2}}
\newcommand{\DiNat}{\mathrm{DiNat}}
\newcommand{\eHom}{\mathbf{hom}}
\newcommand{\VDiNat}[1]{{\bf \DiNat}_{#1}}
\newcommand{\VDiNatZero}[1]{\DiNat_{#1}}
\newcommand{\pqDiNat}[2]{\DiNat^{(#1,#2)}}
\newcommand{\pqNat}[2]{\Nat^{(#1,#2)}}
\def\Tw#1{\mathrm{Tw}(#1)}
\def\defeq{:=}
\usepackage
  { amssymb
  , amsmath
  , adjustbox
  , amsthm
  , booktabs
  , lmodern
  , mathtools
  , tikz-cd
  , todonotes
  , xparse
  , proof
  , wrapfig
  , xspace
  , ytableau
	, cancel
	, commutative-diagrams
}

\usepackage{enumitem}
\DeclareDocumentEnvironment{enumtag}{m}{%
\begin{enumerate}%
	[ label = \textsc{#1}\oldstylenums{\arabic*}),%
		ref   = \textsc{#1}\oldstylenums{\arabic*}%
	] }{ \end{enumerate} }

\newcommand*{\Scale}[2][4]{\scalebox{#1}{\ensuremath{#2}}}%

\newcommand{\pqMat}[1]{
  {\Scale[.925]{\begin{smallmatrix}
    #1 % 1 & 2 \\ 3 & 4
    \end{smallmatrix}}}
}\newcommand{\typepq}[2]{\left[\pqMat{#1\\#2}\right]}

\newcommand{\ConstantP}[2]{\boldsymbol{#1}_{#2}}
\newcommand{\ConstantPNo}[2]{\boldsymbol{#1}_{#2}}

\newcommand{\subst}[3]{{{\boldsymbol{#1}}[#2/#3}]} %#1=list (A1...An), #2=object, #3=number between 1 and n. Substitute #2 at the #3-th entry
\newcommand{\substMV}[4]{{{\boldsymbol{#1}}[#2,#3 / #4]}}

\newcommand{\PSh}[1]{\textsf{\upshape PSh}(#1)}
\newcommand{\PShf}{\textsf{\upshape PSh}}
\newcommand{\Shv}[1]{\textsf{\upshape Sh}(#1)}
\renewcommand{\ell}[1]{\lvert #1 \rvert}
\ExplSyntaxOn
\NewDocumentCommand{\makeabbrev}{mmm}
 {
  \yoruk_makeabbrev:nnn { #1 } { #2 } { #3 }
 }

\cs_new_protected:Npn \yoruk_makeabbrev:nnn #1 #2 #3
 {
  \clist_map_inline:nn { #3 }
   {
    \cs_new_protected:cpn { #2 } { #1 { ##1 } }
   }
 }
\ExplSyntaxOff

\makeabbrev{\textbf}{bf#1}{
  a,b,c,d,e,f,g,h,i,j,k,l,m,n,o,p,q,r,t,u,v,w,x,y,z,%
  A,B,C,D,E,F,G,H,I,J,K,L,M,N,O,P,Q,R,T,U,V,W,X,Y,Z }
\makeabbrev{\boldsymbol}{bs#1}{%
    a,b,c,d,e,f,g,h,i,j,k,l,m,n,o,p,q,r,s,t,u,v,w,x,y,z,%
    A,B,C,D,E,F,G,H,I,J,K,L,M,N,O,P,Q,R,S,T,U,V,W,X,Y,Z }
\makeabbrev{\mathsf}{sf#1}{
  a,b,c,d,e,f,g,h,i,j,k,l,m,n,o,p,q,r,s,t,u,v,w,x,y,z,%
  A,B,C,D,E,F,G,H,I,J,K,L,M,N,O,P,Q,R,S,T,U,V,W,X,Y,Z }
\makeabbrev{\mathfrak}{fk#1}{
  a,b,c,d,e,f,g,h,j,k,i,l,m,n,o,p,q,r,s,t,u,v,w,x,y,z,%
  A,B,C,D,E,F,G,H,I,J,K,L,M,N,O,P,Q,R,S,T,U,V,W,X,Y,Z }
\makeabbrev{\mathcal}{cl#1}{
  A,B,C,D,E,F,G,H,I,J,K,L,M,N,O,P,Q,R,S,T,U,V,W,X,Y,Z }
\makeabbrev{\mathbb}{bb#1}{
  A,B,C,D,E,F,G,H,I,J,K,L,M,N,O,P,Q,R,S,T,U,V,W,X,Y,Z }
\makeabbrev{\underline}{u#1}{
  a,b,c,d,e,f,g,h,j,k,i,l,m,n,o,p,q,r,s,t,u,v,w,x,y,z,%
  A,B,C,D,E,F,G,H,I,J,K,L,M,N,O,P,Q,R,S,T,U,V,W,X,Y,Z }

\def\din{\overset{\bullet\bullet}\Longrightarrow}
\NewDocumentCommand{\pq}{m O{p} O{q}}{
{#1}^{(#2,#3)}
}

\newcommand{\otimesDayN}[1]{\mathbin{\circledast}_{#1}}
\NewDocumentCommand{\wEnd}{o m}{
  \IfNoValueTF{#1}
    {\int_{#2}}
    {\int^{[#1]}_{#2}}
}
\NewDocumentCommand{\wCoend}{o m}{
  \IfNoValueTF{#1}
    {\int^{#2}}
    {\int_{[#1]}^{#2}}
}
\def\Nat{\textsf{Nat}}
\def\yo{y}

\newcommand{\kusarigama}[1]{\Gamma(#1)}
\let\kg\kusarigama
\newcommand{\kgb}[1]{\Gamma\big(#1\big)}
\newcommand{\ikg}[1]{\Gamma^{-1}(#1)}

%
\newcommand{\coKusarigama}[1]{
     \mathchoice%
     {\rotatebox[origin=c]{180}{$\Gamma$}(#1)}
     {\rotatebox[origin=c]{180}{$\Gamma$}(#1)}
     {\rotatebox[origin=c]{180}{$\scriptstyle\Gamma$}(#1)}
     {\rotatebox[origin=c]{180}{$\scriptscriptstyle\Gamma$}(#1)}
}
\let\ckg\coKusarigama
\newcommand{\ckgb}[1]{
     \mathchoice%
     {\rotatebox[origin=c]{180}{$\Gamma$}\big(#1\big)}
     {\rotatebox[origin=c]{180}{$\Gamma$}\big(#1\big)}
     {\rotatebox[origin=c]{180}{$\scriptstyle\Gamma$}\big(#1\big)}
     {\rotatebox[origin=c]{180}{$\scriptscriptstyle\Gamma$}\big(#1\big)}
}
\newcommand{\ickg}[1]{
     \mathchoice%
     {\rotatebox[origin=c]{180}{$\Gamma$}^{-1}(#1)}
     {\rotatebox[origin=c]{180}{$\Gamma$}^{-1}(#1)}
     {\rotatebox[origin=c]{180}{$\scriptstyle\Gamma$}^{-1}(#1)}
     {\rotatebox[origin=c]{180}{$\scriptscriptstyle\Gamma$}^{-1}(#1)}
}
\newcommand{\kgpq}[3]{\Gamma^{#1,#2}(#3)}
\newcommand{\kgpqb}[3]{\Gamma^{#1,#2}\big(#3\big)}
\newcommand{\kgpql}[3]{\Gamma^{#1,#2}\left(#3\right)}
\newcommand{\ckgpq}[3]{
     \mathchoice%
     {\rotatebox[origin=c]{180}{$\Gamma$}^{#1,#2}(#3)}
     {\rotatebox[origin=c]{180}{$\Gamma$}^{#1,#2}(#3)}
     {\rotatebox[origin=c]{180}{$\scriptstyle\Gamma$}^{#1,#2}(#3)}
     {\rotatebox[origin=c]{180}{$\scriptscriptstyle\Gamma$}^{#1,#2}(#3)}
}
\newcommand{\ckgpql}[3]{
     \mathchoice%
     {\rotatebox[origin=c]{180}{$\Gamma$}^{#1,#2}\left(#3\right)}
     {\rotatebox[origin=c]{180}{$\Gamma$}^{#1,#2}\left(#3\right)}
     {\rotatebox[origin=c]{180}{$\scriptstyle\Gamma$}^{#1,#2}\left(#3\right)}
     {\rotatebox[origin=c]{180}{$\scriptscriptstyle\Gamma$}^{#1,#2}\left(#3\right)}
}
\newcommand{\kgfpq}[2]{\Gamma^{#1,#2}}
\newcommand{\ckgfpq}[2]{\rotatebox[origin=c]{180}{$\Gamma$}^{#1,#2}}
%
\newcommand{\kgf}{\Gamma}
\newcommand{\ckgf}{\rotatebox[origin=c]{180}{$\Gamma$}}


\tikzcdset{
    arrow style=tikz,
    diagrams={>={Stealth[round,length=4pt,width=4.95pt,inset=2.75pt]}}
}
\tikzset{
    din/.style={
        double distance=0.2em,
        decoration={
            markings,
            mark={
                at position #1
                with {
                    \draw[fill=white] circle [radius=2pt];
                }
            }
						},
						postaction=decorate
    },
		double,
    din/.default=0em
}


\newcommand{\DiLan}{\mathsf{DiLan}}
\newcommand{\DiRan}{\mathsf{DiRan}}
\newcommand{\VNat}[1]{\mathbf{Nat}_{#1}}
\newcommand{\wLan}[1]{\mathsf{Lan}^{\left[#1\right]}}
\newcommand{\wRan}[1]{\mathsf{Ran}^{\left[#1\right]}}
\newcommand{\wDiLan}[1]{\mathsf{DiLan}^{\left[#1\right]}}
\newcommand{\wDiRan}[1]{\mathsf{DiRan}^{\left[#1\right]}}
\newcommand{\wNat}[1]{\mathsf{Nat}^{[#1]}}
\newcommand{\wDiNat}[1]{\mathsf{DiNat}^{[#1]}}
\newcommand{\wVDiNat}[2]{\mathsf{DiNat}^{[#1]}_{#2}}
\newcommand{\wVNat}[2]{\mathsf{Nat}^{[#1]}_{#2}}
\newcommand{\ev}{\mathsf{ev}}
\newcommand{\Lan}{\mathsf{Lan}}
\newcommand{\Ran}{\mathsf{Ran}}
\newcommand{\M}{\mathsf{M}}
\newcommand{\W}{\mathsf{W}}
\newcommand{\pLan}[3]{\mathsf{Lan}^{#1}_{#2}(#3)}
\newcommand{\pqLan}[4]{\mathsf{Lan}^{(#1,#2)}_{#3}#4}
\newcommand{\pqRan}[4]{\mathsf{Ran}^{(#1,#2)}_{#3}#4}
\newcommand{\pRan}[3]{\mathsf{Ran}^{#1}_{#2}(#3)}

\newcommand{\ceiling}[1]{\lceil#1\rceil}
\let\ceil\ceiling

\usetikzlibrary{decorations.markings}
\tikzset{mid vert/.style={/utils/exec=\tikzset{every node/.append style={outer sep=0.8ex}},
postaction=decorate,decoration={markings,
mark=at position 0.5 with {\draw[-] (0,#1) -- (0,-#1);}}},
mid vert/.default=0.75ex}
\newenvironment{xsmallmatrix}[1]
  {\renewcommand\thickspace{\kern#1}\smallmatrix}
  {\endsmallmatrix}
\NewDocumentCommand{\var}{o m m}{
\IfNoValueTF{#1}{
  \left[
  \begin{smallmatrix} 
  #2 \\
  \downarrow \\ 
  #3
  \end{smallmatrix}\right]}
  {
  \left[
  \begin{xsmallmatrix}{0em}
    & #2 \\ 
    #1 & \downarrow \\ 
    & #3
  \end{xsmallmatrix}\right]}}

\NewDocumentCommand{\dummy}{O{r} O{s}}{
    \text{ð}^{#1}_{\kern-.1em #2}
}

\newcommand{\xloongrightarrow}[1]{%
    \mathrel{%
    \mkern-15.25mu
    \begin{tikzcd}[row sep=3.6em, column sep=2.7em, ampersand replacement=\&]
        {}
        \arrow[r, "#1"] \&
        {}
    \end{tikzcd}
    \mkern-16.25mu
    }
}

\def\colim{\mathrm{colim}}
\NewDocumentCommand{\tpl}{m O{1} O{n}}{
#1_{#2},\dots,#1_{#3}
}

\def\Wd{\mathsf{Wd}}
\def\CWd{\mathsf{CWd}}
\newcommand{\Wedges}[2]{\mathsf{Wd}_{#1}\big(#2\big)}
\newcommand{\Cowedges}[2]{\mathsf{CWd}_{#1}\big(#2\big)}
\newcommand{\pqdiag}{\Delta_{p,q}}
\def\catWd{\mathsf{Wd}}% Category of Wedges
\newcommand{\pqWedges}[4]{\Wd^{(#1,#2)}_{#3}(#4)}
\newcommand{\pqCoWedges}[4]{\CWd^{(#1,#2)}_{#3}(#4)}
\newcommand{\pqWedgesFunctor}[3]{\Wd^{(#1,#2)}_{#3}}
\newcommand{\pqCoWedgesFunctor}[3]{\CWd^{(#1,#2)}_{#3}}
\newcommand{\pqEnd}[3]{\mathop{\prescript{}{\Scale[.75]{\raisebox{.25em}{$(#1,#2)$}}}{\int_{#3}}}}
  %{\int^{(#1,#2)}_{#3}}
\newcommand{\pqTw}[3]{\mathsf{Tw}^{(#1,#2)}(#3)}
\newcommand{\WpqEnd}[4]{\int^{\left[#1\right],(#2,#3)}_{#4}}
\newcommand{\WpqCoend}[4]{\int_{\left[#1\right],(#2,#3)}^{#4}}
% \input{./preamble/kusarigama.tex}
\newcommand{\pqCoend}[3]{%
  \mathchoice
    {\mathop{\prescript{\Scale[.75]{\raisebox{-.25em}{$(#1,#2)$}}\kern-.625em}{}{\int^{#3}}}}
    {\mathop{\prescript{\Scale[.75]{\raisebox{-.25em}{$(#1,#2)$}}\kern-.25em}{}{\int^{#3}}}}
    {\mathop{\prescript{\Scale[.75]{\raisebox{-.25em}{$(#1,#2)$}}\kern-.25em}{}{\int^{#3}}}}
    {\mathop{\prescript{\Scale[.75]{\raisebox{-.25em}{$(#1,#2)$}}\kern-.25em}{}{\int^{#3}}}}
  }

\def\ph{\sfh}
\newcommand{\SloganFont}[1]{{\textit{#1. }}}

\def\Cat{\mathsf{Cat}}
\newtheorem{theorem}{Theorem}[section]
\newtheorem{corollary}{Corollary}
\newtheorem{proposition}{Proposition}
\newtheorem{lemma}{Lemma}
\newtheorem{claim}{Claim}
\newtheorem{definition}{Definition}
\newtheorem{notation}{Notation}
\newtheorem{remark}{Remark}
\newtheorem{conjecture}{Conjecture}
\newtheorem{axiom}{Axiom}
\newtheorem{example}{Example}
\newtheorem{non-example}{Non-Example}
\newtheorem{examples}{Examples}
\newtheorem{exercise}{Exercise}
\newtheorem{counterexample}{Counterexample}
\newtheorem{construction}{Construction}
\newtheorem{warning}{Warning}
\newtheorem{digression}{Digression}
\newtheorem{perspective}{Perspective}
\newtheorem{discussion}{Discussion}
\newtheorem{terminology}{Terminology}
\newtheorem{heuristics}{Heuristics}
\def\op{\mathrm{op}}

\author{Théo de Oliveira Santos$^\ddag$}
\address{%
    \noindent $^\ddag$Universidade de São Paulo,\newline
    Instituto de Ciências Matemáticas e de Computação,\newline
    Av. Trab. São Carlense, 400,\newline
    13566-590 São Carlos, Brasil\newline
    \url{theo.de.oliveira.santos@usp.br}
}

\author{Fosco Loregian$^\S$}
\address{ 
  \noindent $^\S$Tallinn University of Technology,\newline %
  Institute of Cybernetics, Akadeemia tee 15/2, \newline %
  12618 Tallinn, Estonia \newline
  \url{fosco.loregian@taltech.ee}
}

\date{\today}
\title{p-q-coends}
\hypersetup{
 pdfauthor={F and T},
 pdftitle={p-q-coends},
 pdfkeywords={},
 pdfsubject={},
 pdfcreator={Emacs 27.1 (Org mode 9.3.6)}, 
 pdflang={English}}

 \DeclareRobustCommand{\CatEl}[2]{#1\rotatebox[origin=c]{15}{$\int$}#2}
