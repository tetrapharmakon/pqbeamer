\documentclass[11pt]{amsart}
\usepackage[utf8]{inputenc}
\usepackage[T1]{fontenc}
\usepackage{graphicx}
\usepackage{grffile}
\usepackage{longtable}
\usepackage{wrapfig}
\usepackage{rotating}
\usepackage[normalem]{ulem}
\usepackage{amsmath}
\usepackage{textcomp}
\usepackage{amssymb}
\usepackage{capt-of}
\usepackage{hyperref}

\newcommand{\xlongertwoheadrightarrow}[1]{%
    \mathrel{%
        \mkern-15.25mu
        \begin{tikzcd}[row sep=3.6em, column sep=1.8em, ampersand replacement=\&]
            {}
            \arrow[r, "#1", two heads] \&
            {}
        \end{tikzcd}
        \mkern-16.25mu
    }
}
\let\xlongertwoheadsrightarrow\xlongertwoheadrightarrow
% \let\twoheadrightarrow\xlongertwoheadrightarrow

\newcommand{\wlim}[1]{\mathrm{lim}^{#1}}
\newcommand{\wcolim}[1]{\mathrm{colim}^{#1}}
\newcommand{\procirc}{\mathbin{\diamond}}
\newcommand{\ph}{\mathsf{h}}
\newcommand{\WeightedEnd}[2]{\int_{#1}^{#2}}
\newcommand{\WeightedCoend}[2]{\int^{#1}_{#2}}
\newcommand{\DiNat}{\mathrm{DiNat}}
\newcommand{\eHom}{\mathbf{hom}}
\newcommand{\VDiNat}[1]{{\bf \DiNat}_{#1}}
\newcommand{\VDiNatZero}[1]{\DiNat_{#1}}
\newcommand{\pqDiNat}[2]{\DiNat^{(#1,#2)}}
\newcommand{\pqNat}[2]{\Nat^{(#1,#2)}}
\def\Tw#1{\mathrm{Tw}(#1)}
\def\defeq{:=}
\usepackage
  { amssymb
  , amsmath
  , adjustbox
  , amsthm
  , booktabs
  , lmodern
  , mathtools
  , tikz-cd
  , todonotes
  , xparse
  , proof
  , wrapfig
  , xspace
  , ytableau
	, cancel
	, commutative-diagrams
}

\usepackage{enumitem}
\DeclareDocumentEnvironment{enumtag}{m}{%
\begin{enumerate}%
	[ label = \textsc{#1}\oldstylenums{\arabic*}),%
		ref   = \textsc{#1}\oldstylenums{\arabic*}%
	] }{ \end{enumerate} }

\newcommand*{\Scale}[2][4]{\scalebox{#1}{\ensuremath{#2}}}%

\newcommand{\pqMat}[1]{
  {\Scale[.925]{\begin{smallmatrix}
    #1 % 1 & 2 \\ 3 & 4
    \end{smallmatrix}}}
}\newcommand{\typepq}[2]{\left[\pqMat{#1\\#2}\right]}

\newcommand{\ConstantP}[2]{\boldsymbol{#1}_{#2}}
\newcommand{\ConstantPNo}[2]{\boldsymbol{#1}_{#2}}

\newcommand{\subst}[3]{{{\boldsymbol{#1}}[#2/#3}]} %#1=list (A1...An), #2=object, #3=number between 1 and n. Substitute #2 at the #3-th entry
\newcommand{\substMV}[4]{{{\boldsymbol{#1}}[#2,#3 / #4]}}

\newcommand{\PSh}[1]{\textsf{\upshape PSh}(#1)}
\newcommand{\PShf}{\textsf{\upshape PSh}}
\newcommand{\Shv}[1]{\textsf{\upshape Sh}(#1)}
\renewcommand{\ell}[1]{\lvert #1 \rvert}
\ExplSyntaxOn
\NewDocumentCommand{\makeabbrev}{mmm}
 {
  \yoruk_makeabbrev:nnn { #1 } { #2 } { #3 }
 }

\cs_new_protected:Npn \yoruk_makeabbrev:nnn #1 #2 #3
 {
  \clist_map_inline:nn { #3 }
   {
    \cs_new_protected:cpn { #2 } { #1 { ##1 } }
   }
 }
\ExplSyntaxOff

\makeabbrev{\textbf}{bf#1}{
  a,b,c,d,e,f,g,h,i,j,k,l,m,n,o,p,q,r,t,u,v,w,x,y,z,%
  A,B,C,D,E,F,G,H,I,J,K,L,M,N,O,P,Q,R,T,U,V,W,X,Y,Z }
\makeabbrev{\boldsymbol}{bs#1}{%
    a,b,c,d,e,f,g,h,i,j,k,l,m,n,o,p,q,r,s,t,u,v,w,x,y,z,%
    A,B,C,D,E,F,G,H,I,J,K,L,M,N,O,P,Q,R,S,T,U,V,W,X,Y,Z }
\makeabbrev{\mathsf}{sf#1}{
  a,b,c,d,e,f,g,h,i,j,k,l,m,n,o,p,q,r,s,t,u,v,w,x,y,z,%
  A,B,C,D,E,F,G,H,I,J,K,L,M,N,O,P,Q,R,S,T,U,V,W,X,Y,Z }
\makeabbrev{\mathfrak}{fk#1}{
  a,b,c,d,e,f,g,h,j,k,i,l,m,n,o,p,q,r,s,t,u,v,w,x,y,z,%
  A,B,C,D,E,F,G,H,I,J,K,L,M,N,O,P,Q,R,S,T,U,V,W,X,Y,Z }
\makeabbrev{\mathcal}{cl#1}{
  A,B,C,D,E,F,G,H,I,J,K,L,M,N,O,P,Q,R,S,T,U,V,W,X,Y,Z }
\makeabbrev{\mathbb}{bb#1}{
  A,B,C,D,E,F,G,H,I,J,K,L,M,N,O,P,Q,R,S,T,U,V,W,X,Y,Z }
\makeabbrev{\underline}{u#1}{
  a,b,c,d,e,f,g,h,j,k,i,l,m,n,o,p,q,r,s,t,u,v,w,x,y,z,%
  A,B,C,D,E,F,G,H,I,J,K,L,M,N,O,P,Q,R,S,T,U,V,W,X,Y,Z }

\def\din{\overset{\bullet\bullet}\Longrightarrow}
\NewDocumentCommand{\pq}{m O{p} O{q}}{
{#1}^{(#2,#3)}
}

\newcommand{\otimesDayN}[1]{\mathbin{\circledast}_{#1}}
\NewDocumentCommand{\wEnd}{o m}{
  \IfNoValueTF{#1}
    {\int_{#2}}
    {\int^{[#1]}_{#2}}
}
\NewDocumentCommand{\wCoend}{o m}{
  \IfNoValueTF{#1}
    {\int^{#2}}
    {\int_{[#1]}^{#2}}
}
\def\Nat{\textsf{Nat}}
\def\yo{y}

\newcommand{\kusarigama}[1]{\Gamma(#1)}
\let\kg\kusarigama
\newcommand{\kgb}[1]{\Gamma\big(#1\big)}
\newcommand{\ikg}[1]{\Gamma^{-1}(#1)}

%
\newcommand{\coKusarigama}[1]{
     \mathchoice%
     {\rotatebox[origin=c]{180}{$\Gamma$}(#1)}
     {\rotatebox[origin=c]{180}{$\Gamma$}(#1)}
     {\rotatebox[origin=c]{180}{$\scriptstyle\Gamma$}(#1)}
     {\rotatebox[origin=c]{180}{$\scriptscriptstyle\Gamma$}(#1)}
}
\let\ckg\coKusarigama
\newcommand{\ckgb}[1]{
     \mathchoice%
     {\rotatebox[origin=c]{180}{$\Gamma$}\big(#1\big)}
     {\rotatebox[origin=c]{180}{$\Gamma$}\big(#1\big)}
     {\rotatebox[origin=c]{180}{$\scriptstyle\Gamma$}\big(#1\big)}
     {\rotatebox[origin=c]{180}{$\scriptscriptstyle\Gamma$}\big(#1\big)}
}
\newcommand{\ickg}[1]{
     \mathchoice%
     {\rotatebox[origin=c]{180}{$\Gamma$}^{-1}(#1)}
     {\rotatebox[origin=c]{180}{$\Gamma$}^{-1}(#1)}
     {\rotatebox[origin=c]{180}{$\scriptstyle\Gamma$}^{-1}(#1)}
     {\rotatebox[origin=c]{180}{$\scriptscriptstyle\Gamma$}^{-1}(#1)}
}
\newcommand{\kgpq}[3]{\Gamma^{#1,#2}(#3)}
\newcommand{\kgpqb}[3]{\Gamma^{#1,#2}\big(#3\big)}
\newcommand{\kgpql}[3]{\Gamma^{#1,#2}\left(#3\right)}
\newcommand{\ckgpq}[3]{
     \mathchoice%
     {\rotatebox[origin=c]{180}{$\Gamma$}^{#1,#2}(#3)}
     {\rotatebox[origin=c]{180}{$\Gamma$}^{#1,#2}(#3)}
     {\rotatebox[origin=c]{180}{$\scriptstyle\Gamma$}^{#1,#2}(#3)}
     {\rotatebox[origin=c]{180}{$\scriptscriptstyle\Gamma$}^{#1,#2}(#3)}
}
\newcommand{\ckgpql}[3]{
     \mathchoice%
     {\rotatebox[origin=c]{180}{$\Gamma$}^{#1,#2}\left(#3\right)}
     {\rotatebox[origin=c]{180}{$\Gamma$}^{#1,#2}\left(#3\right)}
     {\rotatebox[origin=c]{180}{$\scriptstyle\Gamma$}^{#1,#2}\left(#3\right)}
     {\rotatebox[origin=c]{180}{$\scriptscriptstyle\Gamma$}^{#1,#2}\left(#3\right)}
}
\newcommand{\kgfpq}[2]{\Gamma^{#1,#2}}
\newcommand{\ckgfpq}[2]{\rotatebox[origin=c]{180}{$\Gamma$}^{#1,#2}}
%
\newcommand{\kgf}{\Gamma}
\newcommand{\ckgf}{\rotatebox[origin=c]{180}{$\Gamma$}}


\tikzcdset{
    arrow style=tikz,
    diagrams={>={Stealth[round,length=4pt,width=4.95pt,inset=2.75pt]}}
}
\tikzset{
    din/.style={
        double distance=0.2em,
        decoration={
            markings,
            mark={
                at position #1
                with {
                    \draw[fill=white] circle [radius=2pt];
                }
            }
						},
						postaction=decorate
    },
		double,
    din/.default=0em
}


\newcommand{\DiLan}{\mathsf{DiLan}}
\newcommand{\DiRan}{\mathsf{DiRan}}
\newcommand{\VNat}[1]{\mathbf{Nat}_{#1}}
\newcommand{\wLan}[1]{\mathsf{Lan}^{\left[#1\right]}}
\newcommand{\wRan}[1]{\mathsf{Ran}^{\left[#1\right]}}
\newcommand{\wDiLan}[1]{\mathsf{DiLan}^{\left[#1\right]}}
\newcommand{\wDiRan}[1]{\mathsf{DiRan}^{\left[#1\right]}}
\newcommand{\wNat}[1]{\mathsf{Nat}^{[#1]}}
\newcommand{\wDiNat}[1]{\mathsf{DiNat}^{[#1]}}
\newcommand{\wVDiNat}[2]{\mathsf{DiNat}^{[#1]}_{#2}}
\newcommand{\wVNat}[2]{\mathsf{Nat}^{[#1]}_{#2}}
\newcommand{\ev}{\mathsf{ev}}
\newcommand{\Lan}{\mathsf{Lan}}
\newcommand{\Ran}{\mathsf{Ran}}
\newcommand{\M}{\mathsf{M}}
\newcommand{\W}{\mathsf{W}}
\newcommand{\pLan}[3]{\mathsf{Lan}^{#1}_{#2}(#3)}
\newcommand{\pqLan}[4]{\mathsf{Lan}^{(#1,#2)}_{#3}#4}
\newcommand{\pqRan}[4]{\mathsf{Ran}^{(#1,#2)}_{#3}#4}
\newcommand{\pRan}[3]{\mathsf{Ran}^{#1}_{#2}(#3)}

\newcommand{\ceiling}[1]{\lceil#1\rceil}
\let\ceil\ceiling

\usetikzlibrary{decorations.markings}
\tikzset{mid vert/.style={/utils/exec=\tikzset{every node/.append style={outer sep=0.8ex}},
postaction=decorate,decoration={markings,
mark=at position 0.5 with {\draw[-] (0,#1) -- (0,-#1);}}},
mid vert/.default=0.75ex}
\newenvironment{xsmallmatrix}[1]
  {\renewcommand\thickspace{\kern#1}\smallmatrix}
  {\endsmallmatrix}
\NewDocumentCommand{\var}{o m m}{
\IfNoValueTF{#1}{
  \left[
  \begin{smallmatrix} 
  #2 \\
  \downarrow \\ 
  #3
  \end{smallmatrix}\right]}
  {
  \left[
  \begin{xsmallmatrix}{0em}
    & #2 \\ 
    #1 & \downarrow \\ 
    & #3
  \end{xsmallmatrix}\right]}}

\NewDocumentCommand{\dummy}{O{r} O{s}}{
    \text{ð}^{#1}_{\kern-.1em #2}
}

\newcommand{\xloongrightarrow}[1]{%
    \mathrel{%
    \mkern-15.25mu
    \begin{tikzcd}[row sep=3.6em, column sep=2.7em, ampersand replacement=\&]
        {}
        \arrow[r, "#1"] \&
        {}
    \end{tikzcd}
    \mkern-16.25mu
    }
}

\def\colim{\mathrm{colim}}
\NewDocumentCommand{\tpl}{m O{1} O{n}}{
#1_{#2},\dots,#1_{#3}
}

\def\Wd{\mathsf{Wd}}
\def\CWd{\mathsf{CWd}}
\newcommand{\Wedges}[2]{\mathsf{Wd}_{#1}\big(#2\big)}
\newcommand{\Cowedges}[2]{\mathsf{CWd}_{#1}\big(#2\big)}
\newcommand{\pqdiag}{\Delta_{p,q}}
\def\catWd{\mathsf{Wd}}% Category of Wedges
\newcommand{\pqWedges}[4]{\Wd^{(#1,#2)}_{#3}(#4)}
\newcommand{\pqCoWedges}[4]{\CWd^{(#1,#2)}_{#3}(#4)}
\newcommand{\pqWedgesFunctor}[3]{\Wd^{(#1,#2)}_{#3}}
\newcommand{\pqCoWedgesFunctor}[3]{\CWd^{(#1,#2)}_{#3}}
\newcommand{\pqEnd}[3]{\mathop{\prescript{}{\Scale[.75]{\raisebox{.25em}{$(#1,#2)$}}}{\int_{#3}}}}
  %{\int^{(#1,#2)}_{#3}}
\newcommand{\pqTw}[3]{\mathsf{Tw}^{(#1,#2)}(#3)}
\newcommand{\WpqEnd}[4]{\int^{\left[#1\right],(#2,#3)}_{#4}}
\newcommand{\WpqCoend}[4]{\int_{\left[#1\right],(#2,#3)}^{#4}}
% \input{./preamble/kusarigama.tex}
\newcommand{\pqCoend}[3]{%
  \mathchoice
    {\mathop{\prescript{\Scale[.75]{\raisebox{-.25em}{$(#1,#2)$}}\kern-.625em}{}{\int^{#3}}}}
    {\mathop{\prescript{\Scale[.75]{\raisebox{-.25em}{$(#1,#2)$}}\kern-.25em}{}{\int^{#3}}}}
    {\mathop{\prescript{\Scale[.75]{\raisebox{-.25em}{$(#1,#2)$}}\kern-.25em}{}{\int^{#3}}}}
    {\mathop{\prescript{\Scale[.75]{\raisebox{-.25em}{$(#1,#2)$}}\kern-.25em}{}{\int^{#3}}}}
  }

\def\ph{\sfh}
\newcommand{\SloganFont}[1]{{\textit{#1. }}}

\def\Cat{\mathsf{Cat}}
\newtheorem{theorem}{Theorem}[section]
\newtheorem{corollary}{Corollary}
\newtheorem{proposition}{Proposition}
\newtheorem{lemma}{Lemma}
\newtheorem{claim}{Claim}
\newtheorem{definition}{Definition}
\newtheorem{notation}{Notation}
\newtheorem{remark}{Remark}
\newtheorem{conjecture}{Conjecture}
\newtheorem{axiom}{Axiom}
\newtheorem{example}{Example}
\newtheorem{non-example}{Non-Example}
\newtheorem{examples}{Examples}
\newtheorem{exercise}{Exercise}
\newtheorem{counterexample}{Counterexample}
\newtheorem{construction}{Construction}
\newtheorem{warning}{Warning}
\newtheorem{digression}{Digression}
\newtheorem{perspective}{Perspective}
\newtheorem{discussion}{Discussion}
\newtheorem{terminology}{Terminology}
\newtheorem{heuristics}{Heuristics}
\def\op{\mathrm{op}}

\author{Théo de Oliveira Santos$^\ddag$}
\address{%
    \noindent $^\ddag$Universidade de São Paulo,\newline
    Instituto de Ciências Matemáticas e de Computação,\newline
    Av. Trab. São Carlense, 400,\newline
    13566-590 São Carlos, Brasil\newline
    \url{theo.de.oliveira.santos@usp.br}
}

\author{Fosco Loregian$^\S$}
\address{ 
  \noindent $^\S$Tallinn University of Technology,\newline %
  Institute of Cybernetics, Akadeemia tee 15/2, \newline %
  12618 Tallinn, Estonia \newline
  \url{fosco.loregian@taltech.ee}
}

\date{\today}
\title{p-q-coends}
\hypersetup{
 pdfauthor={F and T},
 pdftitle={p-q-coends},
 pdfkeywords={},
 pdfsubject={},
 pdfcreator={Emacs 27.1 (Org mode 9.3.6)}, 
 pdflang={English}}

 \DeclareRobustCommand{\CatEl}[2]{#1\rotatebox[origin=c]{15}{$\int$}#2}

\begin{document}
\maketitle
\tableofcontents
\section{Motivation:}
\label{sec:org59907ba}
\subsection{``symmetrize'' tensors of higher arity}
\label{sec:orgb9ea8dc}

The so-called ``Einstein summation convention'' in linear algebra and differential geometry 
asserts that it is possible to suppress the summation symbol \(\sum\) in every formula like 
\[
 \textstyle \sum_i c_i v_i
\] at the cost of writing ``\(c^iv_i\)''; this means that contravariant tensors' indices are superscripts,
while covariant tensors' indices are subscripts, and whenever homonymous indices appear in a string 
like \(c^iv_i\), it means that we are summing over that index. So, for example, the first
Bianchi identity:
\[
\nabla_i {R^i}_j = \frac{1}{2}\nabla_j R
\] stands for \(\sum_i \nabla_i {R^i}_j = \dots\), and the definition of \(R\) itself expands to a daunting

\begin{align*}
R_{ij}&=-\sum_{a,b}\frac{1}{2}\Big(\frac{\partial^2g_{ij}}{\partial x^a\partial x^b}+\frac{\partial^2g_{ab}}{\partial x^i\partial x^j}-\frac{\partial^2g_{ib}}{\partial x^j\partial x^a}-\frac{\partial^2g_{jb}}{\partial x^i\partial x^a}\Big)g^{ab}\\
&\qquad+\frac{1}{2}\sum_{a,b,c,d}\Big(\frac{1}{2}\frac{\partial g_{ac}}{\partial x^i}\frac{\partial g_{bd}}{\partial x^j}+\frac{\partial g_{ic}}{\partial x^a}\frac{\partial g_{jd}}{\partial x^b}-\frac{\partial g_{ic}}{\partial x^a}\frac{\partial g_{jb}}{\partial x^d}\Big)g^{ab}g^{cd}\\
&\qquad-\frac{1}{4}\sum_{a,b,c,d}\Big(\frac{\partial g_{jc}}{\partial x^i}+\frac{\partial g_{ic}}{\partial x^j}-\frac{\partial g_{ij}}{\partial x^c}\Big)\Big(2\frac{\partial g_{bd}}{\partial x^a}-\frac{\partial g_{ab}}{\partial x^d}\Big)g^{ab}g^{cd}.
\end{align*}  

This convention does not allow for ``unbalanced'' expression to be summed over: the
same number of subscript must be paired with the same number of superscripts.

In category theory, the analogue operation of ``summing over repeated indices'' is taking a \textbf{coend} of a functor 
\[
T : \mathcal{C}^\text{op} \times \clC \to \clD
\] intended as the quotient  of \(\coprod_C T(C,C)\) by the equivalence relation generated by the action of \(T\) on arrows; this analogy is not peregrine: if \(S : \mathcal{A}^\text{op} \times \mathcal{B} \to \mathsf{Set}\) and \(T : \mathcal{B}^\text{op} \times \clC \to \mathsf{Set}\) are two ``profunctors'', their composition 
\[
ST(A,C) :=  \int^B S(A,B)\times T(B,C)
\] is akin to the matrix product of two matrices, seen as functions \(S : [n]\times [m] \to K\), \(T : [m]\times [r] \to K\).

(A perfect analogy is this: let $A,B,C$ be discrete categories; then the profunctor composition of \(S : A\times B \to \mathsf{Set}\) and \(T : B \times C \to \mathsf{Set}\) is the matrix product of an \(|A|\times |B|\) and a \(|B|\times |C|\) matrix.)

\subsection{Question(s)}
\label{sec:org1c57692}
\begin{itemize}
	\item What if we want to sum/integrate/coend over an ``unbalanced tensor'' like 
	\[
		T : (\clC^\text{op})^p \times \clC^q = \clC^{(p,q)} \to \clD
		\] for \(p,q\ge 1\)?
\item Is the resulting theory well-behaved as the classical one?
\item No one would debate about the usefulness of ``balanced'' integrals; are the unbalanced ones good for something similar?
\end{itemize}
		
		This work aims at answering all these questions in the positive:

\begin{itemize}
\item Yes, one can define a notion of \textbf{co/end} for ``higher arity'' functors \(\clC^{(p,q)} \to \clD\);
\item Yes and no; higher arity co/ends are particular instances of co/ends, where \(T\) has been ``completely symmetrised'' (see later for a definition); as such, they do not constitute a ``new'' object; instead, a specialisation of classical co/end calculus;
\item Yes, the resulting theory is expressive enough to capture some new phenomena.
\end{itemize}

At this point, perhaps the most enlightening example is the following, appearing in a paper by Street and Dubuc: 

\begin{proposition}
Let $F,G : \clC^\text{op} \times \clC \to \clD$ be two functors; define the functor 
\[
\mathrm{DNat}(F^\uparrow,G^\downarrow) : \clC^\text{op} \times \clC \to \mathsf{Set}
\] sending $(A,B)$ to $\clD(F_A^B, G^A_B)$; then, the set of dinatural transformations $F \din G$ is canonically isomorphic to the end of $\text{DNat}(F^\uparrow,G^\downarrow)$, i.e. to the equaliser of the diagram  
\[
\begin{tikzcd}
	\prod_C \clD(F_C^C,G_C^C) \ar[r,shift left, "u"] \ar[r,shift right, "v"'] & \prod_{A\to B} \clD(F_A^B, G_B^A)
\end{tikzcd}	
\]
\end{proposition}

\subsection{generalised dinaturality recently introduced by A. Santamaria in his PhD thesis}
\label{sec:orgc27d079}

A. Santamaria and McGusker recently introduced in [MS] the notion of dinaturality we started from; 
yet, his notion is too general for our purposes for two reasons:

\begin{itemize}
\item they do not assume a ``transformation'' satisfies any condition globally, treating the
\end{itemize}
notion of naturality as a property of a single component (this reads as: a transformation between two functors of a certain ``type'' is di/natural at an index \(i\), but it can be ``unnatural'' elsewhere). 
\begin{itemize}
\item they do not assume that the type of the domain functor and the codomain functor are the same.
\end{itemize}

Our notational convention is also different: they take into account functors \(\clC^\alpha \to \clB\), where \(\alpha\) is a ``binary multi-index'', i.e. an element in the free monoid over the set \(\{\oplus,\ominus\}\), and the convention is that \(\clC^\varnothing := \*\), the terminal category, \(\clC^\oplus := \clC\), \(\clC^\ominus := \clC^\op\), and \(\clC^{\alpha \uplus \alpha'} := \clC^\alpha \times \clC^{\alpha'}\).

Here instead, we adopt a different convention: a generic power \(\clC^\alpha\) is always ``reshuffled'' in order for all its minus and plus signs to appear on the same side, respectively on the left and on the right. The categories \(\clC^\alpha\) and \(\pq{\clC}\) so obtained are, of course, canonically isomorphic, and the tuple \(\alpha\) is equivalent to the reshuffled tuple \((\ominus_1,\dots,\ominus_p,\oplus_1,\dots,\oplus_q)\).

\begin{definition}\label{def:transformation}
	Let $\alpha,\beta$ be two multi-indices, and let $F : \clC^\alpha \to \clD$, $G : \clC^\beta \to \clD$ be functors. A \emph{transformation} $\phi : F \to G$ \emph{of type}
	$
		\begin{tikzcd}[cramped,sep=small]
			\ell\alpha \ar[r,"\sigma"] & n & \ell\beta \ar[l,"\tau"']
		\end{tikzcd}
	$
	(with $n = \ell \bsA$ a positive integer) is a family of morphisms in $\clD$
	\[ 
		\phi_{A_1,\dots,A_n} : F(A_{\sigma 1}, \dots, A_{\sigma\ell\alpha}) \to G(A_{\tau1},\dots,A_{\tau\ell\beta}).
		\]
		for each tuple of objects $\tpl{A}$ of $\clC$.
	\end{definition}

Notice that \(\alpha\) and \(\beta\) are \emph{different} multi-indices in this definition, and \(\sigma,\tau\) need not be injective or surjective, so we may have repeated or unused variables.
\begin{definition}\label{def:dinaturality in i-th variable}
	Let $\phi = (\phi_{A_1,\dots,A_n}) : F \to G$ be a transformation. For $i \in \{1,\dots,n\}$, we say that $\phi$ is \emph{dinatural in $A_i$} (or, more precisely, \emph{dinatural in its $i$-th variable}) if and only if for all $A_1,\dots,A_{i-1}, A_{i+1},\dots,A_n$ objects of $\clC$ and for all $f : A \to B$ in $\clC$ the following hexagon commutes:

	{\footnotesize
	\begin{center}
		\begin{codi}
			\obj [hexagonal=horizontal side 2cm angle 60]
			{
				|(1)| F(\subst \bsA A i \sigma) && |(4)| G(\subst \bsA A i \tau) & \\
				|(2)| F(\substMV \bsA B A i \sigma) &&&  |(5)| G(\substMV \bsA A B i \tau) \\
				|(3)| F(\subst \bsA B i \sigma) && |(6)| G(\subst \bsA B i \tau) & \\
			};
			\mor (2) ["F(\substMV \bsA f A i \sigma)",->] (1) ["\phi_{\subst \bsA A i}",->]  (4) ["G(\substMV \bsA A f i \tau)",->] (5);
			\mor (2) ["F(\substMV \bsA B f i \sigma)"',->] (3) ["\phi_{\subst \bsA B i}"',->] (6) ["G(\substMV \bsA f B i \tau)"',->] (5);
		\end{codi}
	\end{center}
	}
	where $\bsA$ is the $n$-tuple $(A_1,\dots,A_n)$ of the objects above with an additional (unused in this definition) object $A_i$ of $\clC$.
\end{definition}
\section{Higher arity co/wedges}
\begin{enumtag}{n}
	\item\label{n_1} A generic tuple of objects,
	\[\uA := (\tpl{A})\]
	often split as the juxtaposition $\uA';\uA''$ of two subtuples of length $p,q$,
	\[ \uA':= (\tpl{A}[1][q]), \qquad \uA'' :=(\tpl{A}[p+1][p+q])\]
	\item \label{n_2}  As already said, the image of a split tuple $\uA';\uA''$ under a functor of type $\typepq{p}{q}$, $F : \pq{\clC} \to \clD$ is denoted $F^{\uA'}_{\uA''}$: the contravariant components come first, and the covariant component second. So: contravariant components are always \emph{left} in the typing \[F : \pq{\clC} \o \clD\]
	of a functor, and \emph{up} in its action on objects.
	\item \label{n_3}  Denoting a functor $F$ of type $\typepq{p}{q}$ evaluated at a diagonal tuple: we write
	\[F_{\bsA}^{\bsA} := F_{A,\ldots,A}^{A,\ldots,A},\]
	where the superscript has $p$ elements, and the subscript has $q$ elements.
	\item\label{n_4}  Substitution of an object at a prescribed index
	\[\uA[X/i] := (A_1,\dots A_{i-1},X,A_{i+1},\dots, A_n).\]
	\item\label{n_5} Substitution of a tuple at a prescribed tuple of indices
	\[\uA[X_1,\dots,X_r/i_1,\dots,i_r] := ((\uA[X_1/i_1])[X_2/i_2]\cdots)[X_r/i_r].\]
\end{enumtag}
\label{sec:org7927fcb}
\begin{definition}\label{def:p-q-dinatural-transformation}%
	A \emph{$(p,q)$-dinatural transformation} $\alpha: F\din G$ is a collection
	\[
		\big\{\alpha_A: F^{\overset{p\text{ times}}{A,\dots,A}}_{\underset{q\text{ times}}{A,\dots,A}}
		\longrightarrow
		G^{\overset{q\text{ times}}{A,\dots,A}}_{\underset{p\text{ times}}{A,\dots,A}} \mid A \in \clC_o \big\}
	\]
	of morphisms of $\clD$ indexed by the objects of $\clC$ such that, for each morphism $f: A\to  B$ of $\clC$, the diagram
	\begin{center}
		\begin{codi}
			\obj[hexagonal=horizontal side 5.5em angle 60] {%
			|[overwrite=false] (3)| F^{\ConstantPNo{A}{p}}_{\ConstantPNo{A}{q}}
			&
			|[overwrite=false] (2)| G^{\ConstantPNo{A}{q}}_{\ConstantPNo{A}{p}}
			&
			\\
			|[overwrite=false] (4)| F^{\ConstantPNo{B}{p}}_{\ConstantPNo{A}{q}}
			&
			&
			|[overwrite=false] (1)| G^{\ConstantPNo{A}{q}}_{\ConstantPNo{B}{p}}
			\\
			|[overwrite=false] (5)| F^{\ConstantPNo{B}{p}}_{\ConstantPNo{B}{q}}
			&
			|[overwrite=false] (6)| G^{\ConstantPNo{B}{q}}_{\ConstantPNo{B}{p}}
			&
			\\};
			\mor (3) ["\alpha_{A}",->] (2);
			\mor (5) ["\alpha_{B}"',->] (6);
			\mor (4) ["F^{\ConstantPNo{f}{p}}_{\ConstantPNo{A}{q}}",->] (3);
			\mor (4) ["F^{\ConstantPNo{B}{p}}_{\ConstantPNo{f}{q}}"',->] (5);
			\mor (6) ["G^{\ConstantPNo{f}{q}}_{\ConstantPNo{B}{p}}"',->] (1);
			\mor (2) ["G^{\ConstantPNo{A}{q}}_{\ConstantPNo{f}{p}}",->] (1);
		\end{codi}
	\end{center}
	commutes.
\end{definition}
A different name for this same notion: a \emph{$(p,q)$-to-$(q,p)$-dinatural transformation}.
\begin{example}%
	For $(p,q)$=$(2,1)$, a $(2,1)$-dinatural transformation is a collection
	\[
		\Big\{\alpha_{A}: F^{A,A}_{A}\to  G^{A}_{A,A}\ \Big| \ A \in \clC_o\Big\}
	\]
	of morphisms of $\clD$ such that, for each morphism $f: A\to  B$ of $\clC$, the following hexagonal diagram commutes:
	\begin{center}
		\begin{codi}
			\obj[hexagonal=horizontal side 5.0em angle 60] {%
			|[overwrite=false] (3)| F^{A,A}_{A}
			&
			|[overwrite=false] (2)| G^{A}_{A,A}
			&
			\\
			|[overwrite=false] (4)| F^{B,B}_{A}
			&
			&
			|[overwrite=false] (1)| G^{A}_{B,B}
			\\
			|[overwrite=false] (5)| F^{B,B}_{B}
			&
			|[overwrite=false] (6)| G^{B}_{B,B}
			&
			\\};
			\mor (3) ["\alpha_{A}",->] (2);
			\mor (5) ["\alpha_{B}"',->] (6);
			\mor (4) ["F^{f,f}_{A}",->] (3);
			\mor (4) ["F^{B,B}_{f}"',->] (5);
			\mor (6) ["G^{f}_{B,B}"',->] (1);
			\mor (2) ["G^{A}_{f,f}",->] (1);
		\end{codi}
	\end{center}
\end{example}
\subsection{Why the weird ``(p,q)-to-(q,p)'' definition?}
We could have stick to Santamaria's definition of ``$(p,q)$-to-$(r,s)$'' dinaturality; we could have stick to the notion of $(p,q)$-to-$(p,q)$ dinaturality. Our definition sits in the middle:
\begin{quote}
	the type of domain and codomain of a ``higher arity'' dinatural transformation $\alpha : F \din G$ are different, but just swapped: the contravariant length of $F$ is the covariant length of $G$, and vice-versa.
\end{quote}
It is important, even if straightforward, to note that as far as higher arity co/wedges (i.e.\ higher arity dinatural transformations from/to a constant functor) are concerned, the notions of $(p,q)$-dinaturality and $(p,q)$-to-$(r,s)$-dinaturality agree and yield the same theory of higher arity co/ends.

(Recall Mac Lane principle: what is the ``right'' level of generality?)

\begin{definition}\label{def:p-q-wedges}%\leavevmode
	Let $D: \pq{\clC}\longrightarrow\clD$ be a functor and let $X\in\clD_o$.
	\begin{enumtag}{cw}
		\item A \emph{$(p,q)$-wedge for $D$ under $X$} is a $(p,q)$-dinatural transformation $\theta: X\din D$ from the constant functor of type $\typepq{q}{p}$ with value $X$ to $D$;
		\item A \emph{$(p,q)$-cowedge for $D$ over $X$} is a $(p,q)$-dinatural transformation $\zeta: D\din X$ from $D$ to the constant functor of type $\typepq{q}{p}$ with value $X$.
	\end{enumtag}
\end{definition}
\begin{remark}\leavevmode
	\begin{enumtag}{cwu}
		\item\label{unwinding-p-q-wedges}A $(p,q)$-wedge $\theta: X\din D$ is a collection
		\[\big\{\theta_{A}: X\to D^{\bsA}_{\!\bsA}\ : \ A\in\clC_{o}\big\}\]
		of morphisms of $\clC$ such that, for each morphism $f: A\to  B$ of $\clC$, the diagram
		\begin{center}
			\begin{tikzcd}[row sep={4.5em,between origins}, column sep={4.5em,between origins}]%Regular now
				X
				\arrow[r, "\theta_{B}"]
				\arrow[d, "\theta_{A}"']
				&
				D^{\bsB}_{\bsB}
				\arrow[d, "D^{\bsf}_{\bsB}"]
				\\
				D^{\bsA}_{\!\bsA}
				\arrow[r, "D^{\bsA}_{\bsf}"']
				&
				D^{\bsA}_{\bsB}
			\end{tikzcd}
		\end{center}%
		commutes.
		\item\label{unwinding-p-q-cowedges}A $(p,q)$-cowedge $\zeta: D\din X$  is a collection
		\[\big\{\zeta_{A}: D^{\bsA}_{\!\bsA} \to X\ : \ A\in\clC_{o}\big\}\]
		of morphisms of $\clC$ such that, for each morphism $f: A\to  B$ of $\clC$, the diagram
		\begin{center}
			\begin{tikzcd}[row sep={4.5em,between origins}, column sep={4.5em,between origins}]%Regular now
				X
				\arrow[r, "\zeta_{B}",leftarrow]
				\arrow[d, "\zeta_{A}"',leftarrow]
				&
				D^{\bsB}_{\bsB}
				\arrow[d, "D^{\bsf}_{\bsB}",leftarrow]
				\\
				D^{\bsA}_{\!\bsA}
				\arrow[r, "D^{\bsA}_{\bsf}"',leftarrow]
				&
				D^{\bsA}_{\bsB}
			\end{tikzcd}
		\end{center}%
		commutes.
	\end{enumtag}
\end{remark}
\section{Higher arity co/ends}
\begin{definition}\label{def:p-q-ends}
	Let $D:\pq{\clC}\longrightarrow\clD$ be a functor.
	\begin{enumtag}{pq}
		\item\label{pqe}The \emph{$(p,q)$-end} of $D$ is, if it exists, the pair $\smash{\left(\pqEnd{p}{q}{A\in\clC}D^{\uA}_{\!\uA},\omega\right)}$ formed by an object
		\[\pqEnd{p}{q}{A\in\clC}D^{\uA}_{\!\uA}\]
		of $\clD$, and a $(p,q)$-wedge
		\[\omega:{\pqEnd{p}{q}{A\in\clC}D^{\uA}_{\!\uA}}\din D\]
		for ${\pqEnd{p}{q}{A\in\clC}D^{\uA}_{\!\uA}}$ over $D$, such that the $(p,q)$-wedge postcomposition natural transformation
		\[\omega_{*} : \ph\left(-,\pqEnd{p}{q}{A\in\clC}D^{\uA}_{\!\uA}\right)\Longrightarrow\pqWedges{p}{q}{(-)}{D}\]
		is a natural isomorphism.
		\par\vspace*{0.5\baselineskip}
		\item\label{pqc}The \emph{$(p,q)$-coend} of $D$ is, if it exists, the pair $\smash{\left(\pqCoend{p}{q}{A\in\clC}D^{\uA}_{\!\uA},\xi\right)}$ formed by an object
		\[ \pqCoend{p}{q}{A\in\clC}D^{\uA}_{\!\uA} \]
		of $\clD$, and a $(p,q)$-cowedge
		\[ \xi:D\din {\pqCoend{p}{q}{A\in\clC}D^{\uA}_{\!\uA}} \]
		for ${\pqCoend{p}{q}{A}D^{\uA}_{\!\uA}}$ under $D$, such that the $(p,q)$-cowedge postcomposition natural transformation
		\[\xi^{*} : \ph\left(\pqCoend{p}{q}{A\in\clC}D^{\uA}_{\!\uA},-\right)\Longrightarrow\pqCoWedges{p}{q}{(-)}{D}\]
		is a natural isomorphism.
	\end{enumtag}
\end{definition}
\begin{remark}\label{itsa_terminal}
	This means that the $(p,q)$-end of $D$ is the terminal object of the category of wedges of $D$, whose morphisms $h : (\alpha :\Delta_X \din D) \to (\beta :\Delta_Y \din D)$ are defined as the morphisms $h : X \to Y$ of $\clD$ such that for every $A\in \clC_o$ one has $\beta_A \circ h = \alpha_A$:
	\[
		\begin{tikzcd}[row sep={3.6em,between origins}, column sep={3.6em,between origins}]
			X \ar[rr,"h"]\ar[dr, "\alpha_A"']&& Y \ar[dl, "\beta_A"]\\
			& D^{\uA}_{\!\uA}\mathrlap{.}
		\end{tikzcd}
	\]
\end{remark}
\subsection{Basic properties}
\begin{proposition}[Properties of $(p,q)$-ends and $(p,q)$-coends]\label{prop:properties-of-p-q-ends}
	Let $D :\pq{\clC}\longrightarrow\clD$ be a functor.
	\begin{enumtag}{pe}
		\item\label{functoriality-of-p-q-ends}\SloganFont{Functoriality}Let $D :\pq{\clC}\longrightarrow\clD$ be a functor. The assignments $D\mapsto\pqEnd{p}{q}{A}D^{\uA}_{\!\uA},\pqCoend{p}{q}{A}D^{\uA}_{\!\uA}$ define functors
		\begin{align*}
			\pqEnd{p}{q}{A\in\clC}   & : \Cat\big(\pq{\clC},\clD\big) \longrightarrow\clD, \\
			\pqCoend{p}{q}{A\in\clC} & : \Cat\big(\pq{\clC},\clD\big) \longrightarrow\clD
		\end{align*}
		with domain the category of functors from $\clC$ of type $\left[\pqMat{p\\q}\right]$ to $\clD$ and natural transformations between them.
		\item\label{p-q-wedges-and-p-q-diagonals}\SloganFont{$(p,q)$-Wedges and $(p,q)$-diagonals}For each $X\in\clC_o$ we have natural bijections
		\begin{align*}
			\pqWedges{p}{q}{(-)}{D}   & \cong   \Wedges{(-)}{\Delta^{(p,q)}_{*}(D)}, \\
			\pqCoWedges{p}{q}{(-)}{D} & \cong \Cowedges{(-)}{\Delta^{(p,q)}_{*}(D)}.
		\end{align*}
		where $\pqdiag$ is the ``twisted diagonal'' functor% introduced in \cref{def:p-q-diagonal-functor}.
		\[\pqdiag\defeq\underbrace{\Delta^{\op}\times\cdots\times\Delta^{\op}}_{\text{$p$ times}}\times\underbrace{\Delta\times\cdots\times\Delta}_{\text{$q$ times}}.\]
		\item\label{p-q-ends-as-ordinary-ends}\SloganFont{$(p,q)$-Ends as ordinary ends}We have natural isomorphisms
		\begin{align*}
			\pqEnd{p}{q}{A\in\clC}D^{\uA}_{\!\uA}   & \cong \int_{A\in\clC}\Delta^{(p,q)}_{*}(D)^{A}_{A}, \\
			\pqCoend{p}{q}{A\in\clC}D^{\uA}_{\!\uA} & \cong \int^{A\in\clC}\Delta^{(p,q)}_{*}(D)^{A}_{A}.
		\end{align*}
		where $\pqdiag$ is the twisted diagonal functor.% introduced in \cref{def:p-q-diagonal-functor}. 
		In other words, the $(p,q)$-end functor factors as a composition
		\[ 
			\begin{tikzcd}[row sep={11.7em,between origins}, column sep={11.7em,between origins},  ampersand replacement=\&]
				\mathrm{Fun}(\pq{\clC},\clD)
				\arrow[r, "\Delta^{(p,q)}_{*}"]
				% \arrow[rr, to path={%
				% 			|- ([yshift=-2.0ex]\tikztotarget.south)node[near end,below]{$\scriptstyle\pqEnd{p}{q}{A\in\clC}$}
				% 			-- (\tikztotarget)}]
				\&
				\mathrm{Fun}(\clC^{\op}\times\clC,\clD)
				\arrow[r, "\int_{A}"]
				\&
				\clD\mathrlap{,}
			\end{tikzcd}
		\]
		and similarly so do $(p,q)$-coends.
		\item\label{p-q-ends-as-limits}\SloganFont{$(p,q)$-Ends as limits} The $(p,q)$-end and $(p,q)$-coend of $D$ fit respectively into an equaliser and into a coequaliser diagram
		\begin{gather*}
			\begin{tikzcd}[row sep=0cm,ampersand replacement=\&]
				\pqEnd{p}{q}{A\in\clC}D^{\uA}_{\!\uA}
				\arrow[r,outer sep=-0.1em]
				\&
				\displaystyle\prod_{\mathclap{A\in\clC_o}} D^{\uA}_{\!\uA}
				\arrow[r,shift left =1.0, "\lambda"]
				\arrow[r,shift right=1.0, "\rho"']
				\&
				\displaystyle\prod_{\mathclap{A\to B}} D^{\uA}_{\uB}\\
				\& \displaystyle\coprod_{\mathclap{A\to B}}D^{\uA}_{\uB}
				\arrow[r,shift left =1.0,"\lambda'"]
				\arrow[r,shift right=1.0,"\rho'"']
				\&
				\displaystyle\coprod_{\mathclap{A\in\clC_o}}D^{\uA}_{\!\uA}
				\arrow[r,outer sep=-0.1em]
				\&
				\pqCoend{p}{q}{A\in\clC}D^{\uA}_{\!\uA}
			\end{tikzcd}
		\end{gather*}
		for suitable maps $\lambda,\rho,\lambda',\rho'$, induced by the morphisms $D^{\uA}_{\uu}, D^{\uu}_{\uB}$.
		\item\label{p-q-ends-as-limits-again}\SloganFont{$(p,q)$-Ends as limits, again}We have natural isomorphisms
		\begin{align*}
			\pqEnd{p}{q}{A\in\clC}D^{\uA}_{\!\uA}   & \cong \lim\Big(\Tw{\clC}\twoheadrightarrow {\Sigma_{p,q}}\pq{\clC}\xloongrightarrow{D}\clD\Big),   \\
			\pqCoend{p}{q}{A\in\clC}D^{\uA}_{\!\uA} & \cong \colim\Big(\Tw{\clC}\twoheadrightarrow {\Sigma_{p,q}}\pq{\clC}\xloongrightarrow{D}\clD\Big),
		\end{align*}
		where $\Sigma_{p,q}\colon\Tw{\clC}\longrightarrow\pq{\clC}$ is the composition $\Delta^{(p,q)}\circ\Sigma$, with $\Sigma$ the usual Sigma functor from $\Tw{\clC}$ to $\clC^{\op}\times\clC$. Explicitly, $\Sigma^{(p,q)}$ is the functor
		\[
			\begin{tikzcd}[row sep=0cm]
				\Tw{\clC} \ar[r] & \pq{\clC}\\
				\var[f]{A}{B} \ar[r,mapsto] & (\uA,\uB)\\
				\left[
					\begin{smallmatrix}
						& A & \xrightarrow{f}   & B & \\
						\phi\kern-.5em & \uparrow && \downarrow & \kern-.5em\psi \\
						& C & \xrightarrow[g]{} & D
					\end{smallmatrix}
					\right]
				\ar[r,mapsto] & (\underline{\phi},\underline{\psi})
			\end{tikzcd}
		\]
		\item\label{p-q-ends-as-limits-yet-again}\SloganFont{$(p,q)$-Ends as limits, yet again}There exists a category $\pqTw{p}{q}{\clC}$ together with a universal fibration
		\[\Sigma\colon\pqTw{p}{q}{\clC}\twoheadrightarrow\pq{\clC}\]
		inducing natural isomorphisms
		\begin{align*}
			\pqEnd{p}{q}{A\in\clC}D^{\uA}_{\!\uA}   & \cong \lim\Big(\pqTw{p}{q}{\clC}\twoheadrightarrow {\Sigma}\pq{\clC}\xloongrightarrow{D}\clD\Big),   \\
			\pqCoend{p}{q}{A\in\clC}D^{\uA}_{\!\uA} & \cong \colim\Big(\pqTw{p}{q}{\clC}\twoheadrightarrow {\Sigma}\pq{\clC}\xloongrightarrow{D}\clD\Big).
		\end{align*}
		\item\label{p-q-ends-as-p-plus-r-q-plus-s-ends}\SloganFont{$(p,q)$-Ends as $(p+r,q+s)$-ends}we have
		\begin{align*}
			\pqEnd{p}{q}{A\in\clC}D^{\uA}_{\!\uA}   & \cong \pqEnd{p+r}{q+s}{A\in\clC}\dummy(D)^{\uA}_{\uA},   \\
			\pqCoend{p}{q}{A\in\clC}D^{\uA}_{\!\uA} & \cong \pqCoend{p+r}{q+s}{A\in\clC}\dummy(D)^{\uA}_{\uA},
		\end{align*}
		where $\dummy(-)$ is ``$(r,s)$-dummyfication''.% introduced in \cref{not:dummy}.
		\item\label{p-q-ends-commute-with-homs}\SloganFont{Commutativity of $(p,q)$-ends with homs}We have natural isomorphisms
		\begin{align*}
			\clD\left(-,\pqEnd{p}{q}{A\in\clC}D^{\uA}_{\!\uA}\right)   & \cong \pqEnd{p}{q}{A\in\clC}\clD\Big(-,D^{\uA}_{\!\uA}\Big)  \\
			\clD\left(\pqCoend{p}{q}{A\in\clC}D^{\uA}_{\!\uA},-\right) & \cong \pqEnd{q}{p}{A\in\clC}\clD\Big(D^{\uA}_{\!\uA},-\Big).
		\end{align*}
	\end{enumtag}
\end{proposition}
\begin{theorem}[The Fubini Rule]\label{fubini-for-p-q-co-ends}
	Let $D :\clA^{(p,q)}\times\clB^{(r,s)}\longrightarrow\clD$ be a functor. Then
    \begin{gather}
		\pqEnd{p+r}{q+s}{(A,B)}D^{\bsA,\bsB}_{\bsA,\bsB}
		\cong
		\pqEnd{p}{q}{A}\pqEnd{r}{s}{B}D^{\bsA,\bsB}_{\bsA,\bsB}
		\cong
        \pqEnd{r}{s}{B}\pqEnd{p}{q}{A}D^{\bsA,\bsB}_{\bsA,\bsB},\label{fubini-expressions}\\
		\pqCoend{p+r}{q+s}{(A,B)}D^{\bsA,\bsB}_{\bsA,\bsB}
		\cong
		\pqCoend{p}{q}{A}\pqCoend{r}{s}{B}D^{\bsA,\bsB}_{\bsA,\bsB}
		\cong
		\pqCoend{r}{s}{B}\pqCoend{p}{q}{A}D^{\bsA,\bsB}_{\bsA,\bsB}
    \end{gather}
	as objects of $\clD$, meaning that any of these expressions exist if and only if the others do, and, if so, they are are all canonically isomorphic.%
\end{theorem}
\begin{remark}[Fubini does not reduce arity]%
	Note that $p,q,r,s$ can't be broken further: given a functor $G$ of type $\typepq{p}{q}$, its $(p,q)$-end isn't in general expressible in terms of $(p-r,q-s)$-ends for suitable $r,s\ge 1$. This confirms the fact that iterated ends \emph{are not} higher arity ends. Instead, higher arity ends are particular ends. 

    That is, the Fubini rule does not allow us to reduce the arity of a higher arity co/end when $\mathcal{A}=\mathcal{B}$:
	\[
        \pqEnd{p}{q}{A}\pqEnd{r}{s}{B}D^{\bsA,\bsB}_{\bsA,\bsB}
        \cong
		\pqEnd{p+r}{q+s}{(A,B)\in\mathcal{A}\times\mathcal{A}}D^{\bsA,\bsB}_{\bsA,\bsB}
		\ncong
		\pqEnd{p+r}{q+s}{A\in\mathcal{C}}D^{\bsA}_{\bsA}.
	\]
    This is already apparent from the classical Fubini rule, where, given a functor $T\colon\mathcal{C}^{\op}\times\mathcal{C}\times\mathcal{E}^{\op}\times\mathcal{E}\longrightarrow\mathcal{D}$ with $\mathcal{C}=\mathcal{E}$, we have once again
	\[
		\int_{(A,B)\in\mathcal{C}\times\mathcal{C}}T((A,B),(A,B))
		\ncong
		\int_{A\in\mathcal{C}}T(A,A,A,A).
	\]
    The main point in both cases is that we are integrating over a pair $(A,B)$, and not over a single variable $A$.

    From the point of view of adjoints, we have in (e.g.) the $(p,q)=(1,1)$ case
    \begin{align*}
        (-)\odot\left(\ph^{-_{1}}_{-_{3}}\times\ph^{-_{1}}_{-_{4}}\times\ph^{-_{2}}_{-_{3}}\times\ph^{-_{2}}_{-_{4}}\right) &\dashv \pqCoend{2}{2}{A\in\mathcal{C}}D^{A,A}_{A,A}\\
        (-)\odot\underbrace{\ph^{(-_{1},-_{2})}_{(-_{3},-_{4})}}_{\ph^{-_{1}}_{-_{3}}\times\ph^{-_{2}}_{-_{4}}}             &\dashv \int^{(A,B)\in\mathcal{C}\times\mathcal{C}}D^{(A,B)}_{(A,B)},
    \end{align*}
    and of course
    \[
        \ph^{-_{1}}_{-_{3}}\times\ph^{-_{1}}_{-_{4}}\times\ph^{-_{2}}_{-_{3}}\times\ph^{-_{2}}_{-_{4}}
        \neq
        \ph^{(-_{1},-_{2})}_{(-_{3},-_{4})}
        =
        \ph^{-_{1}}_{-_{3}}\times\ph^{-_{2}}_{-_{4}},
    \]
    so $\int^{A\in\mathcal{C}}D^{A,A}_{A,A}$ and $\int^{(A,B)\in\mathcal{C}\times\mathcal{C}}D^{(A,B)}_{(A,B)}$ are different as well.
\end{remark}
\section{Examples:}
\label{sec:orgb57aafa}
\subsection{Some of them are trivial}
\begin{example}[Some $(p,q)$-co/ends are trivial for trivial reasons]\leavevmode
	\begin{itemize} 
		\item The $(0,2)$-ends and $(0,2)$-coends of the functor $\otimes : \clC \times \clC \to \clC$ that gives $\clC$ a monoidal structure are trivial under very mild assumptions on $\clC$. This rules out a class of possibly interesting examples coming from multilinear algebra.

	Consider the category $\mathsf{Mod}_R$ (with additional care one can take left modules and right modules, of course). To show that
	\begin{align*}
		\pqEnd{0}{2}{A}{A\otimes A}   & \cong \lim_{A,B\in\mathsf{Mod}_R} A\otimes B    \cong 0, \\
		\pqCoend{0}{2}{A}{A\otimes A} & \cong \colim_{A,B\in\mathsf{Mod}_R} A\otimes B  \cong 0,
	\end{align*}
	we just observe that $\mathsf{Mod}_R$ is a sifted category, because it admits finite coproducts. The fact that a category $\clC$ is sifted if and only if $\clC$ is non-empty and the diagonal functor $\Delta_{\clC}\colon\clC\longrightarrow\clC\times\clC$ is final%
	%--- Begin Footnote ---%
	\footnote{%
		This is due to \cite{gabriel-ulmer}; see also \cite[Proposition 5.3.2]{richter2020categories} or \cite[Theorem 2.15]{algebraic-theories-arv} for reviews.
	} %
	%---  End Footnote  ---%
	then yields the result.

	\item If $\clC$ is a sifted category, all diagonal functors $\Delta : \clC \to \clC^n$ are final, because the product and composition of final functors is itself final. Thus the same result transports to higher coends of higher arity functors: for example, $\bigwedge^{k} : \mathsf{Mod}_R^{n} \to \mathsf{Mod}_R$ sending $M$ to $\bigwedge^{k}M$.
	\item Let $R$ be a ring. The \emph{walking cochain complex} (\cite[Paragraph 35.1]{joyal:quasi}) is the $\mathsf{Mod}_{R}$-enriched category $\textsf{Ch}$ whose set of objects is the set of integers, and where the hom-sets are $R$-modules freely generated by
	\[
		\sfC([m],[n])=\begin{cases}
			\{d,0\} & \text{ if } m=n+1\text{,} \\
			\{1,0\}   & \text{ if } m=n\text{,}   \\
			\{0\}     & \text{ otherwise.}
		\end{cases}
	\]
	Now, a cochain complex is precisely a $\mathsf{Mod}_{R}$-enriched functor from the $\textsf{Ch}$ to $\mathsf{Mod}_{R}$. Similarly, \emph{bi}complexes are $\mathsf{Mod}_{R}$-enriched functors from $\textsf{Ch}\boxtimes_{\mathsf{Mod}_{R}}\textsf{Ch}$ to $\mathsf{Mod}_{R}$.

	Let $D\colon\textsf{Ch}\boxtimes_{\mathsf{Mod}_{R}}\textsf{Ch}\longrightarrow\mathsf{Mod}_{R}$ be a bicomplex. We claim that its $\mathsf{Mod}_{R}$-enriched $(2,0)$-end $E$ is just the zero module. Indeed, looking at $(2,0)$-wedges, we see that they are either of the form
	\begin{center}
		\begin{tikzcd}[row sep=2.7em, column sep=2.7em,  ampersand replacement=\&]
			E
			\arrow[r]
			\arrow[d]
			\&
			D^{n,n}
			\\
			D^{n,n}
			\arrow[ru,equal]
			\&
		\end{tikzcd}
		\quad
		or of the form
		\quad
		\begin{tikzcd}[row sep=2.7em, column sep=2.7em,  ampersand replacement=\&]
			E
			\arrow[r]
			\arrow[d]
			\&
			D^{n+1,n+1}
			\\
			D^{n,n}
			\arrow[ru,"\mathrm{d}^{n,n}"']
			\&
		\end{tikzcd}
	\end{center}%
	Now, it follows from the second diagram that
	\[E\cong\textstyle \big\{(a_{k})_{k\in\bbZ}\in\prod_{k\in\bbZ}D^{k,k}\ \mid\ a_{k+1}=\mathrm{d}^{k,k}(a_{k})\big\},\]
	but differentials square to zero, so we must have $a_{k}=0$ for all $k\in\bbZ$, and $E$ is indeed isomorphic to the zero module. A similar argument shows that $\pqCoend{0}{2}{[k]\in\textsf{Ch}}D^{k,k}\cong0$.
\end{itemize}
\end{example}
\begin{example}[Bisimplicial sets]
	Recall that a \emph{bisimplicial set} (\cite[Chapter IV]{goerss-jardine}, \cite[\S 3.1.15]{cisinski-model-categories}) is a functor $X : \Delta^\op\times \Delta \to \mathsf{Set}$; moreover, the \emph{diagonalisation} of a bisimplicial set $X_{\bullet,\bullet}\colon\Delta^\op\times\Delta^\op\longrightarrow\mathsf{Set}$ is the simplicial set $\text{d}(X)_{\bullet}\colon\Delta^\op\longrightarrow\mathsf{Set}$ given by
	\[\text{d}(X)_{n}\defeq X_{n,n}.\]
	Joining the products and equalisers formula for $(p,q)$-coends we  see that $\pqCoend{2}{0}{[n]\in\Delta^\op}X_{n,n}$ is the coequaliser of he diagram
	\[
		\begin{tikzcd}[cramped,ampersand replacement=\&]
			\displaystyle\coprod_{\mathclap{[n]\to[m]\in\Delta}}X_{m,m},
			\arrow[r,shift left =1.0]
			\arrow[r,shift right=1.0]
			\&
			\displaystyle\coprod_{\mathclap{[n]\in\Delta}}X_{n,n}
		\end{tikzcd}
	\]
	giving
	\[\pqCoend{2}{0}{[n]\in\Delta}X_{n,n}\cong\pi_{0}(\text{d}(X)).\]
	By a similar argument, we have
	\[\pqEnd{2}{0}{[n]\in\Delta}X_{n,n}\cong X_{0,0}.\]
\end{example}
\label{sec:orgc0066fa}
\subsection{Juicy examples:}
\subsubsection{A glance at weighted co/ends}\label{glance_at_weightends}
Weighted co/ends stand to co/ends in the same relation as weighted co/limits stand to limits.
\begin{definition}[Weighted co/end]\label{def:weighted-co-ends}
	Let $\clC$ and $\clD$ be $\clV$-enriched categories and $D\colon\clC^{\op}\otimes_{\clV}\clC\longrightarrow\clD$ a $\clV$-functor, and $W :\clC^\op\times \clC \to \clV$ a $\clV$-presheaf.
	\begin{enumtag}{we}
		\item The \emph{end of $D$ weighted by $W$} is, if it exists, the object $\smash{\WeightedEnd{A\in\clC}{W}D^{A}_{A}}$ of $\clD$ with the property that
		\[\hom_{\clD}\Big(X,{\WeightedEnd{A\in\clC}{W}D^{A}_{A}}\Big)\cong\VDiNat{\clV}(W,\eHom_{\clC}(X,D))\]
		naturally in $X\in\clD$.
		\item The \emph{coend of $D$ weighted by $W$} is, if it exists, the object $\smash{\WeightedCoend{A\in\clC}{W}D^{A}_{A}}$ of $\clD$ with the property that
		\[\hom_{\clD}\Big({\WeightedCoend{A\in\clC}{W}D^{A}_{A}},Y\Big)\cong\VDiNat{\clV}(W,\eHom_{\clC}(D,Y))\]
		naturally in $Y\in\clD$.
	\end{enumtag}
\end{definition}
\begin{example}[Weighted co/ends are $(2,2)$-co/ends]
	A quick argument (to be discussed in future work \cite{weighend}) gives $(2,2)$-co/end formulas for weighted co/ends:
	\begin{align*}
		\wEnd[W]{A\in\clC}D^{A}_{A}   & \cong \pqEnd{2}{2}{A\in\clC}W^{A}_{A}\pitchfork D^{A}_{A}, \\
		\wCoend[W]{A\in\clC}D^{A}_{A} & \cong \pqCoend{2}{2}{A\in\clC}W^{A}_{A}\odot D^{A}_{A}.
	\end{align*}
\end{example}
\begin{example}[Weighting Increases Arity]
	Let $F,G\colon\clC\longrightarrow\clD$ and $W\colon\clC^{\op}\times\clC\longrightarrow\clV$ be $\clV$-functors. In analogy with
	\[
		\VNat{\clV}(F,G)
		\defeq
		\int_{A\in\clC}\eHom_{\clD}(F_{A},G_{A}),
	\]
	we define the \emph{object $\wNat{W}(F,G)$ of natural transformations from $F$ to $G$ weighted by $W$} by
	\begin{equation}\label{object-wnat-w-f-g-of-natural-transformations-from-f-to-g-weighted-by-w}
		\wNat{W}(F,G)
		\defeq
		\wEnd[W]{A\in\clC}\eHom_{\clD}(F_{A},G_{A}).
	\end{equation}
	Taking $W$ to be mute in its contravariant variable, we can give a reformulation of the universal property of weighted limits:
	\[
		\ph\left(-,\wlim{W}(D)\right)
		\cong
		\wNat{W}\left(\Delta_{(-)},D\right).
	\]
	Defining $\wVDiNat{W}{\clV}(F,G)$ by a similar formula, we also obtain the following isomorphism in the case of weighted ends:
	%Similarly, in analogy to
	%\[
	%	\DiNat(F,G)
	%	\cong
	%	\int_{A\in\clC}\hom_{\clD}\left(F^{A}_{A},G^{A}_{A}\right),
	%\]
	%you define
	%\[
	%	\wDiNat{W}(F,G)
	%	\cong
	%	\wEnd[W]{A\in\clC}\hom_{\clD}\left(F^{A}_{A},G^{A}_{A}\right),
	%\]
	%which (circularly) gives:
	\[
		\ph\left(-,\wEnd[W]{A\in\clC}D^{A}_{A}\right)
		\cong
		\wVDiNat{W}{\clV}\left(\Delta_{(-)},D\right).
	\]
	This naturally suggests a definition of \emph{doubly-weighted ends}:
	\[
		\ph\left(-,\wEnd[W_{1},W_{2}]{A\in\clC}D^{A}_{A}\right)
		\cong
		\wVDiNat{W_{1}}{\clV}(W_{2},D).
	\]
	Repeating this process give you ends weighted by a collection of $n$ functors $W_{1},\ldots,W_{n}$. These however, can be actually computed as $(n+1,n+1)$-ends (\cite{weighend}):
	\[
		\int_{A\in\clC}^{[W_{1},\ldots,W_{n}]}D^{A}_{A}
		\cong
		\pqEnd{n+1}{n+1}{A\in\clC}\left((W_{1})^{A}_{A}\times\cdots\times(W_{n})^{A}_{A}\right)\odot D^{A}_{A}.
	\]
	As such, we see that weighting an end increases its arity by $(1,1)$.
\end{example}
\subsubsection{Weighted Kan extensions}\label{weikan}
Another source of examples comes from ``weighing'' left and right Kan extensions. While the most general such weight is a profunctor, having type $\typepq{1}{1}$, weights of type $\typepq{1}{0}$ or $\typepq{0}{1}$ are specially interesting, as they give a more direct parallel with the classical theory of weighted co/limits.

Recall the definition of the object $\wNat{W}(F,G)$ of weighted natural transformations.
\begin{definition}\label{left-weikan}
	The \emph{left Kan extension of $F$ along $K$ weighted by $W$} is, if it exists, the $\clV$-functor
	\[
		\left(\wLan{W}_{K}F\colon\clD\longrightarrow\clE\right)\colon
		\begin{tikzcd}[row sep={3.6em,between origins}, column sep={3.6em,between origins}, ampersand replacement=\&]
			{}
			\&
			\clD
			\arrow[d, "\wLan{W}_{K}F", dashed]
			\\
			\clC
			\arrow[ru, "K"]
			\arrow[l, "W",mid vert,loop left]
			\arrow[r, "F"'{name=F}]
			\&
			\clE
			% 2-Arrows
			\arrow[from=F,to=1-2,shorten=1.125em,Rightarrow,xshift=-0.125em,yshift=-0.25em]
		\end{tikzcd}
	\]
	for which we have a $\clV$-natural isomorphism
	\begin{equation}\label{weighted-left-kan-extension-defining-equation}
		\VNat{\clV}\left(\wLan{W}_{K}F,G\right)\cong\wVNat{W}{\clV}\left(F,G\circ K\right),
	\end{equation}
	natural in $G$.
\end{definition}
One defines weighted right Kan extensions in a dual manner:
\begin{definition}\label{right-weikan}
	The \emph{right Kan extension of $F$ along $K$ weighted by $W$} is, if it exists, the $\clV$-functor
	\[
		\left(\wRan{W}_{K}F\colon\clD\longrightarrow\clE\right)\colon
		\begin{tikzcd}[row sep={3.6em,between origins}, column sep={3.6em,between origins}, ampersand replacement=\&]
			{}
			\&
			\clD
			\arrow[d, "\wRan{W}_{K}F", dashed]
			\\
			\clC
			\arrow[ru, "K"]
			\arrow[l, "W",mid vert,loop left]
			\arrow[r, "F"'{name=F}]
			\&
			\clE
			% 2-Arrows
			\arrow[from=F,to=1-2,shorten=1.125em,Leftarrow,xshift=-0.125em,yshift=-0.25em]
		\end{tikzcd}
	\]
	for which we have a $\clV$-natural isomorphism
	\begin{equation}\label{weighted-right-kan-extension-defining-equation}
		\VNat{\clV}\left(G,\wRan{W}_{K}F\right)\cong\wVNat{W}{\clV}\left(G\circ K,F\right),
	\end{equation}
	natural in $G$.
\end{definition}
\begin{example}[Weighted co/limits as weighted Kan extensions]\label{ex:weighted-co-limits-weighted-kan-extensions}
	Let $D\colon\clC\longrightarrow\clD$ be a diagram on a category $\clD$. Then we may canonically identify the left Kan extension of $D$ along the terminal functor with its colimit:
	\[
		\Lan_{!}D\cong\ceil{\colim(D)}
		\quad
		\begin{tikzcd}[row sep={3.6em,between origins}, column sep={3.6em,between origins}, ampersand replacement=\&]
			{}
			\&
			1
			\arrow[d, "\ceil{\colim(D)}", dashed]
			\\
			\clC
			\arrow[ru, "!"]
			%\arrow[l, "W",mid vert,loop left]
			\arrow[r, "D"'{name=F}]
			\&
			\clD
			% 2-Arrows
			\arrow[from=F,to=1-2,shorten=1.0em,Rightarrow,xshift=-0.125em,yshift=-0.25em]
		\end{tikzcd}
	\]
	Similarly, given a weight $W\colon\clC^{\op}\longrightarrow\mathsf{Set}$, we have
	\[
		\wLan{W}_{!}D\cong\ceil{\wcolim{W}(D)}
		\quad
		\begin{tikzcd}[row sep={3.6em,between origins}, column sep={3.6em,between origins}, ampersand replacement=\&]
			{}
			\&
			1
			\arrow[d, "\ceil{\wcolim{W}(D)}", dashed]
			\\
			\clC
			\arrow[ru, "!"]
			\arrow[l, "W",mid vert,loop left]
			\arrow[r, "D"'{name=F}]
			\&
			\clD
			% 2-Arrows
			\arrow[from=F,to=1-2,shorten=1.0em,Rightarrow,xshift=-0.125em,yshift=-0.25em]
		\end{tikzcd}
	\]
\end{example}
One can also prove that the following formulas hold (\cite{weighend}):
\begin{align}
	\wLan{W}_{K}F & \cong \wCoend[W]{A\in\clC}\eHom_{\clC}\left(K_{A},-\right)     \odot     F_{A} \cong \pqCoend{2}{2}{A\in\clC}\left(W^{A}_{A}\times\eHom_{\clC}(K_{A},-)\right)\odot      F_{A}, \label{eq:weighted-kan-extension-formula-left} \\
	\wRan{W}_{K}F & \cong \wEnd[W]{A\in\clC}  \eHom_{\clC}\left(-,K_{A}\right)\pitchfork F_{A} \cong \pqEnd{2}{2}{A\in\clC}  \left(W^{A}_{A}\times\eHom_{\clC}(-,K_{A})\right)\pitchfork F_{A}.\label{eq:weighted-kan-extension-formula-right}
\end{align}
Equipped with these, we now proceed to compute a few weighted Kan extensions.
\begin{example}
	Consider the functor $\imath^\op : 1^\op \to \Delta^\op$;
	the left and right Kan extensions of a set $X_{\bullet}\colon1\longrightarrow\mathsf{Set}$ along $\imath^{\op}$ are given by
	\begin{align*}
		\Lan_{\imath^{\op}}(X) & \cong \underline{X}_{\bullet} \\
		\Ran_{\imath^{\op}}(X) & \cong \check{C}(X).
	\end{align*}
	Now take a weight $W\colon1^{\op}\times1\longrightarrow \mathsf{Set}$:
	\[
		\begin{tikzcd}[row sep={4.5em,between origins}, column sep={4.5em,between origins}, ampersand replacement=\&]
			{}
			\&
			\Delta^\op
		\arrow[d, "\wLan{W}_{\imath^{\op}}X", dashed]
			\\
			1^{\op}
			\arrow[ru, "\imath^{\op}"]
			\arrow[l, "W",mid vert,loop left]
			\arrow[r, "X"'{name=F}]
			\&
			\mathsf{Set}
			% 2-Arrows
			% \arrow[from=F,to=1-2,shorten=1.125em,Rightarrow,xshift=-0.125em,yshift=-0.25em]
		\end{tikzcd}
	\]
	Then
	\begin{align*}
		\wLan{W}_{\imath^{\op}}(X) & \cong \underline{W\times X}_{\bullet} \\
		\wRan{W}_{\imath^{\op}}(X) & \cong \check{C}(W\times X).
	\end{align*}
\end{example}
\begin{example}
	Now for the more interesting counterpart of the above:
	\[
		\begin{tikzcd}[row sep=0.0em, column sep=2.7em,  ampersand replacement=\&]
			 \pi^{\op}\colon \Delta^{\op}
			\arrow[r]
			\&
			1^{\op}
			\\
			{[n]}
			\arrow[r, mapsto]
			\&
			{\star}
		\end{tikzcd}
	\]%
	The left and right Kan extensions of a simplicial set $X_{\bullet}\colon\Delta^\op\longrightarrow\mathsf{Set}$ along $\pi^{\op}$ are given by
	\begin{align*}
		\Lan_{\pi^{\op}}(X_{\bullet}) & \cong \pi_{0}(X_{\bullet})               \\
		\Ran_{\pi^{\op}}(X_{\bullet}) & \cong \ev_{0}(X_{\bullet}) \defeq X_{0}.
	\end{align*}
	Now, a weight $W^{\bullet}_{\bullet}\colon\Delta^{\op}\times\Delta\longrightarrow\mathsf{Set}$ is wonderfully complicated: it is a cosimplicial space!
	\[
		% \begin{tikzcd}[row sep={4.5em,between origins}, column sep={4.5em,between origins}, ampersand replacement=\&]
		% 	{}
		% 	\&
		% 	1^{\op}
		% 	\arrow[d, "\wLan{W}_{\pi^{\op}}X_{\bullet}", dashed]
		% 	\\
		% 	\Delta^\op
		% 	\arrow[ru, "\pi^{\op}"]
		% 	\arrow[l, "W",mid vert,loop left]
		% 	\arrow[r, "X_{\bullet}"'{name=F}]
		% 	\&
		% 	\mathsf{Set}
		% 	% 2-Arrows
		% 	\arrow[from=F,o=1-2,shorten=1.125em,Rightarrow,xshift=-0.125em,yshift=-0.25em]
		% \end{tikzcd}
	\]
	Then
	\begin{enumerate}
		\item Taking $W=\Delta^{\bullet}$ almost gives the geometric realisation of $X_{\bullet}$:
		      \[
			      \wLan{\Delta^{\bullet}}_{\pi^{\op}}(X_{\bullet}) \cong \int^{[n]\in\Delta}\Delta^{n}\times X_{n}.
		      \]
		\item Dually, taking again $W=\Delta^{\bullet}$ but now a cosimplicial object $X^{\bullet}\colon\Delta\longrightarrow\mathsf{Set}$,
		      \[
			      \wRan{\Delta^{\bullet}}_{\pi}(X^{\bullet}) = Tot(X_{\bullet}).
		      \]
		\item If ou take $W=\Delta^{\bullet}_{\bullet}=\hom_{\Delta}(-,-)$, then I think you get
		      \begin{align*}
			      \wLan{\Delta^{\bullet}_{\bullet}}_{\pi^{\op}}(X_{\bullet}) & \cong \pqCoend{2}{2}{[n]\in\Delta}\Delta^{n}_{n}\times X_{n} \\
			                                                                 & \cong \pqCoend{2}{2}{[n]\in\Delta}\Delta^{n}_{n}\times X_{n}
		      \end{align*}
	\end{enumerate}
\end{example}
\begin{example}
	Using the fact that weighted left/right Kan extensions along the identity are adjoint to each other, we can study situations like
	\[
		\begin{tikzcd}[row sep={4.5em,between origins}, column sep={4.5em,between origins}, ampersand replacement=\&]
			{}
			\&
			\Delta^\op
			\arrow[d, "?", dashed]
			\\
			\Delta^\op
			\arrow[ru, equal, "1"]
			\arrow[l, "W^{\bullet}_{\bullet}",mid vert,loop left]
			\arrow[r, "X_{\bullet}"'{name=F}]
			\&
			\mathsf{Set}
			% 2-Arrows
			\arrow[from=F,to=1-2,shorten=1.25em,Rightarrow,xshift=-0.125em,yshift=-0.25em]
		\end{tikzcd}
	\]
	This gives rise  to an adjunction $L : \textsf{sSet}\leftrightarrows\textsf{sSet} : R$ with
	\begin{align*}
		L(X_{\bullet}) & \cong \pqEnd{2}{2}{  [n]\in\Delta}W^{n}_{n}\odot      X_{n} \cong \pqEnd{2}{2}{  [n]\in\Delta} W^{n}_{n}\times X_{n},  \\
		R(X_{\bullet}) & \cong \pqCoend{2}{2}{[n]\in\Delta}W^{n}_{n}\pitchfork X_{n} \cong \pqCoend{2}{2}{[n]\in\Delta}[W^{n}_{n},      X_{n}].
	\end{align*}
	Taking $W=\Delta^{\bullet}$ gives $L=R=1$, so let's take something more complicated, like $\Delta^{\bullet}_{\bullet}$. Then
	\begin{align*}
		L(X_{\bullet}) & \cong \pqEnd{2}{2}{  [n]\in\Delta} \Delta^{n}[n]\times X_{n}  \cong ? \\
		R(X_{\bullet}) & \cong \pqCoend{2}{2}{[n]\in\Delta}[\Delta^{n}[n],      X_{n}] \cong ?
	\end{align*}
\end{example}
\begin{example}[Weighing the stalks of a sheaf]% {(\cite[Paragraph 6.8 and Section 7.1]{sgaiv})}]
	Let $i_{p}\colon\{p\}\hookrightarrow X$ be the inclusion of a point into a topological space $X$. We get an induced functor
	\[
		\begin{tikzcd}[row sep=0.0em, column sep=2.7em,  ampersand replacement=\&]
			\clO(i_{p})\colon \clO(X)
			\arrow[r]
			\&
			\clO(\{p\})
			\\
			U
			\arrow[r, mapsto]
			\&
			i_{p}^{-1}(U)
		\end{tikzcd}
	\]%
	Considering now left Kan extensions along the opposite of $\clO(i_{p})$,
	\[
		\begin{tikzcd}[row sep={6.3em,between origins}, column sep={6.3em,between origins}, ampersand replacement=\&]
			{}
			\&
			\clO(\{p\})^{\op}
			\arrow[d, "\Lan_{\clO(i_{p})^{\op}}\mathcal{F}", dashed]
			\\
			\clO(X)^{\op}
			\arrow[ru, "\clO(i_{p})^{\op}"]
			\arrow[r, "\mathcal{F}"'{name=F}]
			\&
			\mathsf{Set}
			% 2-Arrows
			\arrow[from=F,to=1-2,shorten=1.75em,Rightarrow,xshift=-0.125em,yshift=-0.25em]
		\end{tikzcd}
	\]
	we obtain a functor $\Lan_{\clO(i_{p})^{\op}}\colon\PSh{X}\longrightarrow\PSh{\{p\}}$, whose image at $\mathcal{F}$ is written $\ceil{\mathcal{F}_{p}}$ for simplicity. The restriction of this functor to $\Shv{X}$ can be identified with the stalk functor $(-)_{p}\colon\Shv{X}\longrightarrow\mathsf{Set}$: we have $\clO(\{p\})=\{\varnothing\hookrightarrow\{p\}\}$ and computing the images of $\varnothing$ and $\{p\}$ under $\ceil{\mathcal{F}_{p}}$ via the usual colimit formula for left Kan extensions gives
	\begin{align*}
		\ceil{\mathcal{F}_{p}}(\{p\})       & \cong \colim\left(\left(\clO(\ceil{p})\downarrow\underline{\{p\}}\right)^{\op}\xlongertwoheadrightarrow{\pi^{\op}}\clO(X)^{\op}\xrightarrow{\mathcal{F}}\mathsf{Set}\right),       \\
		                                      & \cong \colim_{U\ni p}(\mathcal{F}(U)),                                                                                                                                                   \\
		                                      & \cong \mathcal{F}_{p}                                                                                                                                                                    \\
		\ceil{\mathcal{F}_{p}}(\varnothing) & \cong \colim\left(\left(\clO(\ceil{p})\downarrow\underline{\varnothing}\right)^{\op}\xlongertwoheadrightarrow{\pi^{\op}}\clO(X)^{\op}\xrightarrow{\mathcal{F}}\mathsf{Set}\right), \\
		                                      & \cong \colim_{U\hookrightarrow\varnothing}(\mathcal{F}(U)),                                                                                                                              \\
		                                      & \cong \mathcal{F}(\varnothing).
	\end{align*}
	(in case $\mathcal{F}$ is a sheaf, $\mathcal{F}(\varnothing)$ is the singleton set.) Consider the same situation, but now with a weight $W\colon\clO(X)\times\clO(X)^{\op}\longrightarrow\mathsf{Set}$ (an "extradiagonal presheaf on $X$"):
	\[
		\begin{tikzcd}[%[row sep={6.3em,between origins}, column sep={6.3em,between origins}, 
			ampersand replacement=\&]
			{}
			\&
			\clO(\{p\})^{\op}
			\arrow[d, "\wLan{W}_{\clO(i_{p})^{\op}}\mathcal{F}", dashed]
			\\
			\clO(X)^{\op}
			\arrow[ru, "\clO(i_{p})^{\op}"]
			\arrow[l, "W",mid vert,loop left]
			\arrow[r, "\mathcal{F}"'{name=F}]
			\&
			\mathsf{Set}
			% 2-Arrows
			\arrow[from=F,to=1-2,shorten=1em,Rightarrow,xshift=-0.125em,yshift=-0.25em]
		\end{tikzcd}
	\]
	We may compute $\wLan{W}_{\clO(i_{p})^{\op}}\mathcal{F}\defeq\ceil{\mathcal{F}_{p}^{[W]}}$ as the weighted coend
	\begin{align*}
		\ceil{\mathcal{F}^{[W]}_{p}} & \defeq \wCoend[W]{U\in\clO(X)}\hom_{\clO(X)^{\op}}\big(\clO\big(i_{p}^{\op}\big)(U),-\big)\\odotmathcal{F}(U) \\
		                               & \cong  \wCoend{U\in\clO(X)}W^{U}_{U}\times\hom_{\clO(X)}\big(\chi_{p}(U),-)\big)\\timesmathcal{F}(U),
	\end{align*}
	where
	\[
		\chi_{p}(U)
		=
		\begin{cases}
			\emptyset & \text{if $p\notin U$,} \\
			U         & \text{otherwise.}
		\end{cases}
	\]
	For instance, taking $W$ to be a sheaf $\mathcal{G}$ on $X$ gives
	\[
		\mathcal{F}^{[\mathcal{G}]}_{p} \defeq \ceil{\mathcal{F}^{[\mathcal{G}]}_{p}}(\{p\}) \cong \big(\mathcal{F}\times\mathcal{G}\big)_{p}.
	\]
\end{example}
\subsubsection{A glance at extradiagonality}\label{glance_at_extradiag}
``Extradiagonal'' category theory arises when, instead of considering a natural transformation filling a higher-dimensional cell, we consider a \emph{dinatural} one. Transformations that are more general than natural ones notoriously do not compose; yet, the category theory arising from this generalisation is interesting.

\begin{definition}[Diagonal left Kan extensions]\label{def:diagonal-left-kan-extensions}%
	The \emph{diagonal left Kan extension} of a functor $F\colon\clC^{\op}\times\clC\longrightarrow\clD$ along a functor $K\colon\clC^{\op}\times\clC\longrightarrow\clD$ is, if it exists the functor $\DiLan_{K}F\colon\clD\longrightarrow\clE$ such that we have an isomorphism
	\[
		\Nat(\DiLan_{K}F,G)
		\cong
		\DiNat(F,G\circ K)
		\qquad
		\begin{tikzcd}[row sep={5.4em,between origins}, column sep={5.4em,between origins}, ampersand replacement=\&]
			{}
			\&
			\clD
			\arrow[d, "\DiLan_{K}F", dashed]
			\\
			\clC^{\op}\times\clC
			\arrow[ru, "K"]
			\arrow[r, "F"'{name=F}]
			\&
			\clE
			% 2-Arrows
			\arrow[from=F,to=1-2,din=0.60,shorten <=1.2em,shorten >=1.5em+0.35em]
		\end{tikzcd}
	\]
	natural in $G$.
\end{definition}
\begin{example}\label{ex:coends-dilan}
	Standard examples of diagonal left Kan extensions are ends: Generalising the fact that the left Kan extension of a functor $D\colon\clC\longrightarrow\clD$ along the terminal functor $\pi\colon\clC\twoheadrightarrow 1$ can be identified with the colimit of $\clD$, the diagonal left Kan extension of a functor $D\colon\clC^{\op}\times\clC\longrightarrow\clD$ along the terminal functor $\pi\colon\clC^{\op}\times\clC\twoheadrightarrow 1$ can be identified with the coend of $\clD$.
	\[
		\begin{tikzcd}[row sep={4.5em,between origins}, column sep={4.5em,between origins}, ampersand replacement=\&,cramped]
			{}
			\&
			1
			\arrow[d, "\ceil{\colim(D)}", dashed]
			\\
			\clC
			\arrow[ru, "\pi"]
			\arrow[r, "D"'{name=F}]
			\&
			\clD
			% 2-Arrows
			\arrow[from=F,to=1-2,Rightarrow,shorten=1.25em,xshift=0.1em]
		\end{tikzcd}
		\begin{tikzcd}[row sep={4.5em,between origins}, column sep={4.5em,between origins}, ampersand replacement=\&,cramped]
			{}
			\&
			1
			\arrow[d, "\ceil{\int^{A}D^{A}_{A}}", dashed]
			\\
			\clC^{\op}\times\clC
			\arrow[ru, "\pi"]
			\arrow[r, "D"'{name=F}]
			\&
			\clD
			% 2-Arrows
			\arrow[from=F,to=1-2,din=0.65,shorten <=1.0em,shorten >=1.0em+0.3em,xshift=-0.1em]
		\end{tikzcd}
	\]
\end{example}
Now, while ordinary Kan extensions can be computed via co/end formulas, diagonal Kan extensions admit $(2,2)$-co/end formulas (\cite{extradiag}):
\begin{align}
	\DiLan_{K}F & \cong \pqCoend{2}{2}{A\in\clC}\clD\left(K^{A}_{A},-\right)\odot F^{A}_{A},   \label{eq:dilankf} \\
	\DiRan_{K}F & \cong \pqEnd{2}{2}{A\in\clC}\clD\left(-,K^{A}_{A}\right)\pitchfork F^{A}_{A},\label{eq:dirankf}
\end{align}
where the pairing is such that $\DiLan_{K}F$ is the coend of
\[
	(A,B)\mapsto\clD\left(K^{B}_{A},-\right)\odot F^{A}_{B}.
\]
Alternatively, we may compute diagonal Kan extensions as $\hom$-weighted Kan extensions (\cite{weighend,extradiag}):
\begin{align*}
	\DiLan_{K}F & \cong \wCoend[\hom_{\clC}(-,-)]{A,B\in\clC}\clD\left(K^{B}_{A},-\right)\odot      F^{A}_{B}, \\
	\DiRan_{K}F & \cong \wEnd[\hom_{\clC}(-,-)]{A,B\in\clC}  \clD\left(-,K^{B}_{A}\right)\pitchfork F^{A}_{B}.
\end{align*}
This is a generalisation of the fact that ends are $\hom$-weighted limits.
A forthcoming work \cite{extradiag} will address the topic of this remark in its entirety, studying the category theory arising from the notion of a weighted co/end.
\begin{example}
	Let $\clC$ be a closed monoidal category and $D\colon\clC^{\op}\times\clC\longrightarrow\clD$ be a diagram on $\clD$. What is $\DiLan_{[-,-]}D$ and $\DiRan_{[-,-]}D$?
	\[
		\begin{tikzcd}[row sep={5.4em,between origins}, column sep={5.4em,between origins}, ampersand replacement=\&]
			{}
			\&
			\clC
			\arrow[d, "\DiLan_{[-,-]}D", dashed]
			\\
			\clC^{\op}\times\clC
			\arrow[ru, "{[-,-]}"]
			\arrow[r, "D"'{name=F}]
			\&
			\clD
			% 2-Arrows
			\arrow[from=F,to=1-2,din=0.625,shorten <=1.6em,shorten >=1.6em+0.3em,xshift=-0.125em]
		\end{tikzcd}
	\]
	\[
		\DiLan_{[-,-]}D \cong \int^{A\in\clC}\hom_{\clC}([A,A],-)\odot D^{A}_{A}.
	\]
\end{example}
\begin{example}
	Let $D\colon\clC^{\op}\times\clC\longrightarrow\clD$ be a diagram on $\clD$. What is $\DiLan_{\yo}D$ and $\DiRan_{\yo}D$?
	\[
		\begin{tikzcd}[row sep={5.4em,between origins}, column sep={5.4em,between origins}, ampersand replacement=\&]
			{}
			\&
			\PSh{\clC^{\op}\times\clC}
			\arrow[d, "\DiLan_{\yo}D", dashed]
			\\
			\clC^{\op}\times\clC
			\arrow[ru, "{\yo}",hook]
			\arrow[r, "D"'{name=F}]
			\&
			\clD
			% 2-Arrows
			\arrow[from=F,to=1-2,din=0.625,shorten <=1.6em,shorten >=1.6em+0.3em,xshift=-0.125em]
		\end{tikzcd}
	\]
	\begin{align*}
		\DiLan_{\yo}D & \cong  \int^{A,B\in\clC}_{\hom_{\clC}(-,-)}\hom_{\PSh{\clC^{\op}\times\clC}}\left(\yo^{B}_{A},-\right)\odot F^{A}_{B},              \\
		              & \cong  \int^{A\in\clC}\hom_{\PSh{\clC^{\op}\times\clC}}\left(\yo^{A}_{A},-\right)\odot F^{A}_{A},                                   \\
		              & \cong  \pqCoend{2}{2}{A\in\clC}\hom_{\PSh{\clC^{\op}\times\clC}}\left(\yo^{A}_{A},-\right)\odot F^{A}_{A},                          \\
		              & \defeq \pqCoend{2}{2}{A\in\clC}\hom_{\PSh{\clC^{\op}\times\clC}}\left(\hom_{\clC^{\op}\times\clC}(-,(A,A)),-\right)\odot F^{A}_{A}, \\
		              & \defeq \pqCoend{2}{2}{A\in\clC}\hom_{\PSh{\clC^{\op}\times\clC}}\left(\ph^{A}\times\ph_{A},-\right)\odot F^{A}_{A}.
	\end{align*}
\end{example}
In order to introduce the next example, we recall the following notation: we have an adjunction
\[
	(\pi\dashv\imath)\colon
	\begin{tikzcd}[row sep=2.7em, column sep=2.7em,  ampersand replacement=\&]
		1
		\arrow[r, "\imath"'{name=F}, hook, shift right = 1.8, bend right = 0] \&
		\Delta\mathrlap{,}
		\arrow[l, "\pi"'{name=G}, two heads, shift right = 1.8, bend left=0]
		%--- Adjunction Symbol
		\arrow[phantom, from=F, to=G, "\dashv" rotate=-90]
	\end{tikzcd}
\]
where
\begin{itemize}
	\item $\imath\colon1\hookrightarrow\Delta$ is the functor choosing the terminal object;
	\item $\pi\colon\Delta\twoheadrightarrow 1$ is the terminal functor;
\end{itemize}
This induces a quadruple adjunction%
\[
	\left(\pi_0\dashv\underline{(-)}_{\bullet}\dashv\ev_{0}\dashv\check{C}\right)\colon
	\begin{tikzcd}[
			ampersand replacement=\&]
		\mathsf{Set}
		\ar[r, shift left=.5em]\ar[r,shift right=.5em]\&
		\textsf{sSet}
		\ar[l, "\pi_0"', shift right=1em]
		\ar[l]
	\end{tikzcd}
\]
\begin{example}
	Let $S_{\bullet}^{\bullet}\colon\Delta^\op\times\Delta\longrightarrow\mathsf{Set}$ be a cosimplicial space. What is $\DiLan_{\pi^{\op}\times\pi}(S_{\bullet}^{\bullet})$?
	\[
		\begin{tikzcd}[row sep={5.4em,between origins}, column sep={5.4em,between origins}, ampersand replacement=\&]
			{}
			\&
			1^{\op}\times1
			\arrow[d, "\DiLan_{\pi^{\op}\times\pi}(S^{\bullet}_{\bullet})", dashed]
			\\
			\Delta^\op\times\Delta
			\arrow[ru, "{\pi^{\op}\times\pi}",two heads]
			\arrow[r, "S^{\bullet}_{\bullet}"'{name=F}]
			\&
			\mathsf{Set}
			% 2-Arrows
			\arrow[from=F,to=1-2,din=0.625,shorten <=1.6em,shorten >=1.6em+0.3em,xshift=-0.125em]
		\end{tikzcd}
	\]
	It is just the end of $S^{\bullet}_{\bullet}$ (btw do you know what this is?):
	\begin{align*}
		\DiLan_{\pi^{\op}\times\pi}(S^{\bullet}_{\bullet}) & \cong \int^{[n]\in\Delta}\hom_{1}(\star,\star)\odot S^{n}_{n}, \\
		                                                   & \cong \int^{[n]\in\Delta}S^{n}_{n}.
	\end{align*}
	Similarly, given a set $X\colon1^{\op}\times1\longrightarrow\mathsf{Set}$, we have
	\[
		\begin{tikzcd}[row sep={5.4em,between origins}, column sep={5.4em,between origins}, ampersand replacement=\&]
			{}
			\&
			\Delta^\op\times\Delta
			\arrow[d, "\DiLan_{\imath^{\op}\times\imath}(X)", dashed]
			\\
			1^{\op}\times1
			\arrow[ru, "{\imath^{\op}\times\imath}",hook]
			\arrow[r, "X"'{name=F}]
			\&
			\mathsf{Set}
			% 2-Arrows
			\arrow[from=F,to=1-2,din=0.625,shorten <=1.6em,shorten >=1.6em+0.3em,xshift=-0.125em]
		\end{tikzcd}
	\]
	\begin{align*}
		\DiLan_{\imath^{\op}\times\imath}(X) & \cong \int^{\star\in1}\hom_{\Delta^\op\times\Delta}(([0],[0]),(-_{1},-_{2}))\odot X, \\
		                                 & \cong \hom_{\Delta^\op\times\Delta}(([0],[0]),(-_{1},-_{2}))\odot X                       \\
		                                 & \cong \hom_{\Delta}([0],-_{2})\odot X                                                               \\
		                                 & \cong \Delta^{-_{2}}[0]\odot X.
	\end{align*}
	Similarly, let $X_{\bullet}^{\bullet}\colon\Delta^\op\times\Delta\longrightarrow\mathsf{Set}$ be a cosimplicial space again. What is $\DiLan_{\Delta}(X_{\bullet}^{\bullet})$?
	\[
		\begin{tikzcd}[row sep={5.4em,between origins}, column sep={5.4em,between origins}, ampersand replacement=\&]
			{}
			\&
			\mathsf{Set}
			\arrow[d, "\DiLan_{\Delta}(X^{\bullet}_{\bullet})", dashed]
			\\
			\Delta^\op\times\Delta
			\arrow[ru, "{\Delta^{-_{2}}[-_{1}]}"]
			\arrow[r, "X^{\bullet}_{\bullet}"'{name=F}]
			\&
			\mathsf{Set}
			% 2-Arrows
			\arrow[from=F,to=1-2,din=0.625,shorten <=1.6em,shorten >=1.6em+0.3em,xshift=-0.125em]
		\end{tikzcd}
	\]
	\[
		\DiLan_{\Delta}(X^{\bullet}_{\bullet}) \cong \int^{[n]\in\Delta}\mathsf{Set}(\Delta^{n}[n],-)\odot X^{n}_{n}.
	\]
\end{example}
\subsubsection{Weighted diagonal Kan extensions}
In the same spirit, one can define weighted diagonal Kan extensions, mixing the two perspectives and considering now the diagram
\[
	\begin{tikzcd}[row sep={5.4em,between origins}, column sep={5.4em,between origins}, ampersand replacement=\&]
		{}
		\&
		\clD
		\arrow[d, "\DiLan_{K}F", dashed]
		\\
		\clC^{\op}\times\clC
		\arrow[l, "W",mid vert,loop left]
		\arrow[ru, "K"]
		\arrow[r, "F"'{name=F}]
		\&
		\clE
		% 2-Arrows
		\arrow[from=F,to=1-2,din=0.65,shorten <=1.3em,shorten >=1.3em+0.3em]
	\end{tikzcd}
\]
just to discover that these are actually computed as $(4,4)$-co/ends:
\begin{align*}
	\wDiLan{W}_{K}F & \cong \pqCoend{4}{4}{A\in\clC}\left(W^{A,A}_{A,A}\times\eHom_{\clC}\left(K^{A}_{A},-\right)\right)\odot      F^{A}_{A}, \\
	\wDiRan{W}_{K}F & \cong \pqEnd{4}{4}{A\in\clC}\left(W^{A,A}_{A,A}\times  \eHom_{\clC}\left(-,K^{A}_{A}\right)\right)\pitchfork F^{A}_{A}.
\end{align*}
At this point, the reader shall be convinced that the list of examples is virtually endless. % hahaha
We defer a thorough study of the topic to separate works \cite{extradiag,weighend}.
\subsubsection{Daydreaming About Operads}
Day convolution was introduced by B.\ Day in \cite{day:thesis,day:report}, in order to classify monoidal structures on the category $\PSh{\clC}$ of presheaves on $\clC$. Day proved that $\PSh{\clC}$ can be turned into a monoidal category in as many ways as $\clC$ can be turned into a pseudomonoid in the bicategory of profunctors.%\footnote{More formally, let $S : \Cat \to \Cat$ be the 2-monad of pseudomonoids; let $\tilde S : \Prof \to \Prof$ be the lifting of $S$ to the bicategory of profunctors (i.e.\ to the Kleisli bicategory of the presheaf construction $\PShf$); then, given an object $\clC$ of $\Cat$, there is a bijection between pseudo-$S$-algebra structures on $\PSh{\clC}$ and pseudo-$\tilde S$-algebras on $\clC$, as an object of $\Prof$.}

We now propose a generalisation of this framework based on higher arity coends: let $(\clC,\otimes,I)$ be a monoidal category, and let $\clK\defeq \PSh{\clC}$. Higher arity Day convolution is defined as a family of functors $\otimesDayN{n} : \clK^n\to \clK$:
\begin{definition}\label{nn_day_convolution}
	The \textbf{Day $(n,n)$-convolution} of an $n$-tuple of presheaves $\tpl{\mathcal{F}}$ is the presheaf
	\[\otimesDayN{n}(\tpl{\mathcal{F}})\colon\clC^{\op}\longrightarrow\mathsf{Set}\]
	defined at $A\in\clC_o$ as the $(n,n)$-coend
	\[\otimesDayN{n}(\tpl{\mathcal{F}}) \defeq A \mapsto \pqCoend{n}{n}{A\in\clC}\mathcal{F}_{1}(A)\times\cdots\times\mathcal{F}_{n}(A)\times\clC\left(-,A^{\otimes n}\right),\]
	where $A^{\otimes n}$ is shorthand for the $n$-fold tensor product of $A$ with itself.
\end{definition}
\begin{example}[Day convolution operad]\label{the-day-higher-arity-convolution-operad}%
	The \textbf{Day convolution operad associated to $(\clC,\otimes,I)$} is the free symmetric operad $\mathsf{Day}$ whose set of generating operations (see \cite[Section 1.2.5]{fresse-operads}) is given by $\{1,\otimesDayN{2},\otimesDayN{3},\ldots,\otimesDayN{n},\ldots\}$.
\end{example}
\begin{remark}%[Unwinding \cref{the-day-higher-arity-convolution-operad}]%
	We spell out in detail the first four sets of $n$-ary operations of $\mathsf{Day}$:
	\begin{align*}
		\mathsf{Day}_{1} & = \{1\}                                                                                                                                                            \\
		\mathsf{Day}_{2} & = \{\otimesDayN{2}(-,-)\}                                                                                                                                            \\
		\mathsf{Day}_{3} & = \{\otimesDayN{3}(-,-,-),\otimesDayN{2}(\otimesDayN{2}(-,-),-),\otimesDayN{2}(-,\otimesDayN{2}(-,-))\}                                                              \\
		\mathsf{Day}_{4} & = \{\otimesDayN{4}(-,-,-,-),\otimesDayN{2}(\otimesDayN{3}(-,-,-),-),\otimesDayN{2}(-,\otimesDayN{3}(-,-,-)),\otimesDayN{2}(\otimesDayN{2}(-,-),\otimesDayN{2}(-,-)), \\\
		               & \phantom{=\{} \otimesDayN{3}(-,\otimesDayN{2}(-,-),-),\otimesDayN{3}(\otimesDayN{2}(-,-),-,-),\otimesDayN{3}(-,-,\otimesDayN{2}(-,-))\}
	\end{align*}
	All in all, the set $\mathsf{Day}_n$ can be succinctly described as
	\[\mathsf{Day}_n = \{\otimesDayN{n}\} \cup \sum_{p+q=n} \mathsf{Day}_p \times \mathsf{Day}_q \]
	The operadic composition of $\mathsf{Day}$ is now defined via `grafting' in the usual way:
	\[ 
		\begin{tikzcd}[row sep=0.0em, column sep=2.7em,  ampersand replacement=\&]
			\mathsf{Day}_{n}\times\mathsf{Day}_{k_{1}}\times\cdots\times\mathsf{Day}_{k_{n}}
			\arrow[r]
			\&
			\mathsf{Day}_{\sum k_i}
			\\
			{(\theta;\theta_{1},\ldots,\theta_{k})}
			\arrow[r, mapsto]
			\&
			{\theta(\theta_{1}(-_{1},\ldots,-_{k_{1}}),\ldots,\theta_{k}(-_{1},\ldots,-_{k_{n}}))}
		\end{tikzcd}
		\]
\end{remark}
\section{Kusarigamas}
\label{sec:orgade2373}
\emph{Kusarigamas} are functors of type $\typepq{q}{p}$ attached to a functor $G$ of type $\typepq{p}{q}$, and enjoying a universal property among these. They are functors 
\begin{align*}
	\ckgfpq{p}{q} & : \Cat\big(\pq{\clC},\clD\big)\longrightarrow\Cat\big(\pq{\clC}[q][p],\clD\big), \\
	\kgfpq{p}{q}  & : \Cat\big(\pq{\clC},\clD\big)\longrightarrow\Cat\big(\pq{\clC}[q][p],\clD\big), 
\end{align*}
that can be regarded as
\begin{itemize}
	\item Universal objects among $(p,q)$-dinatural transformations, through which all other $(p,q)$-dinaturals factor:% (\cref{def:co-kusarigamas,rem:unwinding-co-kusarigamas}):
	      \[ \pqDiNat{p}{q}\left(F,G\right)                 \cong \Nat\left(F,\kgpq{p}{q}{G}\right)             \cong \Nat\left(\ckgpq{p}{q}{F},G\right);\]
	      % \item A certain {\color{red} comonad-monad} adjunction $\ckgfpq{p}{q}\dashv \kgfpq{p}{q} : \Cat(\pq{\clC}, \clD) \leftrightarrows \Cat(\pq{\clC}, \clD)$ of a special kind (\cref{missing}), based on the `iterated tracing indentities' of \cref{missing};
	\item Functors that can be inductively defined through suitable Kan extensions% (\cref{higher-arity-co-kusarigamas-from-1-1-co-kusarigamas}), 
	starting from the case $\typepq{1}{1}$:
	      \[ \kgpq{p}{q}{G} \cong \Ran_{\pqdiag^*G}{\kgpq{1}{1}{G}}; \qquad \ckgpq{p}{q}{F} \cong \Lan_{\pqdiag^* F}{\ckgpq{1}{1}{F}}.\]
\end{itemize}
The paramount property of the co/kusarigama functors is that 
\begin{quote}
	given a category $\clC$, the category of elements of $\ckgpq{p}{q}{1}$, where $1 : \pq{\clC} \to \textsf{Set}$ is the terminal presheaf, is the universal fibration needed to build a higher-arity version of the \emph{twisted arrow category} (i.e., the category of elements of $\hom_\clC$).
\end{quote}
This makes it possible to express the $(p,q)$-co/end of $G : \pq{\clC} \to \clD$ as a co/limit over the $(p,q)$-twisted arrow category of $\clC$: 
\begin{align*}
	\pqEnd{p}{q}{A\in\clC}D^{\uA}_{\uA}   & \cong \lim\Big(\pqTw{p}{q}{\clC}\xloongrightarrow{\Sigma_{(p,q)}}\pq{\clC}\xloongrightarrow{D}\clD\Big),               \\
	\pqCoend{p}{q}{A\in\clC}D^{\uA}_{\uA} & \cong \colim\Big(\pqTw{p}{q}{\clC^{\op}}^{\op}\xloongrightarrow{\Sigma_{(p,q)}}\pq{\clC}\xloongrightarrow{D}\clD\Big).
\end{align*}
\subsection{Reducing dinaturality to naturality, and other properties}
\begin{construction}[Constructing co/kusarigamas]\leavevmode\label{con:constructing-cokusarigamas}%
	\begin{itemize} \item Suppose that $\clD$ is cocomplete. Then
	\[
		\pqCoend{p}{q}{A\in\clC}\left(\ph^{-}_{\bsA_{q}}\times\ph^{\bsA_{p}}_{-}\right)\odot F^{\bsA_{p}}_{\bsA_{q}}
	\]%
	meaning the $(p,q)$-coend of
	\[
		\begin{tikzcd}[row sep=0.0em, column sep=2.7em, ampersand replacement=\&]
			\pq{\clC}
			\arrow[r]
			\&
			\Cat(\pq{\clC}[q][p],\clD)
			\\
			(\uA,\uB)
			\arrow[r, mapsto]
			\&
			\hom_{\pq{\clC}[q][p]}\left((\uB,\uA);(-,-)\right)\odot F^{\uA}_{\uB},
		\end{tikzcd}
	\]
	is the cokusarigama of $F$.
	\item Suppose that $\clD$ is complete. Then
	\[
		\pqEnd{q}{p}{A\in\clC}\left(\ph^{\bsA_{p}}_{-}\times\ph^{-}_{\bsA_{q}}\right)\pitchfork G^{\bsA_{q}}_{\bsA_{p}},
	\]%
	meaning the $(q,p)$-coend of
	\[%begin{diagram}\label{diagram:constructions-of-kusarigamas}
		\begin{tikzcd}[row sep=0.0em, column sep=2.7em, ampersand replacement=\&]
			\pq{\clC}[q][p]
			\arrow[r]
			\&
			\Cat(\pq{\clC},\clD)
			\\
			(\uA,\uB)
			\arrow[r, mapsto]
			\&
			\hom_{\pq{\clC}[q][p]}\left((\uA,\uB);(-,-)\right)\pitchfork G^{\uA}_{\uB},
		\end{tikzcd}
	\]%end{diagram}
	is the kusarigama of $G$.
\end{itemize}
\end{construction}
Explicitly, 
\begin{gather*}
\kgfpq{p}{q}(G)(\uX,\uY)  \cong \pqEnd{q}{p}{A\in\clC} \Big(\ph^{A}_{X_1} \times \dots \times\ph^{A}_{X_p} \times \ph^{Y_1}_{A} \times \dots \times\ph^{Y_q}_A \Big)\pitchfork G^{A,\dots,A}_{A,\dots,A}\\
\ckgfpq{p}{q}(F)(\uX,\uY)  \cong \pqCoend{q}{p}{A\in\clC} \Big(\ph^{A}_{X_1} \times \dots \times \ph^{A}_{X_p} \times \ph^{Y_1}_{A} \times \dots \times\ph^{Y_q}_A \Big)\odot F^{A,\dots,A}_{A,\dots,A}\\
\end{gather*}
\begin{proposition}[Properties of Co/kusarigamas]\label{properties-of-co-kusarigamas}
	Let $D,F,G :\pq{\clC}\rightrightarrows\clD$ be functors, where $\clD$ is a bicomplete category.
	\begin{enumtag}{pk}
		\item\label{adjointness-of-co-kusarigamas}\SloganFont{Adjointness}We have an adjunction
		\begin{equation*}
			\begin{tikzcd}
				\Cat\big(\pq{\clC},\clD\big)
				\arrow[r, "\ckgfpq{p}{q}"{name=F}, shift left=1.75] &
				\Cat\left(\pq{\clC}[q][p],\clD\right).
				\arrow[l, "\kgfpq{q}{p}"{name=G}, shift left=1.75]
				\arrow[phantom, from=F, to=G, "\dashv" rotate=-90]
			\end{tikzcd}
		\end{equation*}
		\item\label{commutativity-of-co-kusarigamas-with-homs}\SloganFont{Commutativity with homs} Let $F : \pq{\clC} \to \clD$ be a functor, and let us consider the functors
		\begin{align*}
            \clD(F,1) : \clD \to \Cat(\pq{\clC}[q][p],\textsf{Set}), D     & \mapsto \big((\uA,\uB)\mapsto\clD\left(F^{\uA}_{\uB}, D\right)\big),\\
            \clD(1,F) : \clD^\op \to \Cat(\pq{\clC},\textsf{Set}), D       & \mapsto \big((\uA,\uB)\mapsto\clD\left(D,F^{\uA}_{\uB}\right)\big),
		\end{align*}
		% $h^D$ be the functor $\clD(D,1) : \clD \to \PSh{\pq{\clC}[q][p]}$, and similarly for $h_D$; 
		then the diagrams
		\begin{center}
			\begin{adjustbox}{max height=0.5\textheight, max width=0.875\textwidth}
				\parbox{\linewidth}{
					\[
                        \begin{tikzcd}[row sep={4.5em,between origins}, column sep={5.0em,between origins}, ampersand replacement=\&]
							\&\clD\ar[dr, "{\clD(F,1)}"]\ar[dl, "{\clD(\ckgpq{p}{q}F,1)}"']\& \\
							\Cat(\pq{\clC}[q][p],\textsf{Set}) \ar[rr,"\kgfpq{q}{p}"']\&\& \Cat(\pq{\clC},\textsf{Set})
						\end{tikzcd}\qquad
                        \begin{tikzcd}[row sep={4.5em,between origins}, column sep={5.0em,between origins}, ampersand replacement=\&]
							\&\clD \ar[dr, "{\clD(1,F)}"]\ar[dl, "{\clD(1,\kgpq{p}{q}F)}"'] \& \\
							\Cat(\pq{\clC},\textsf{Set}) \ar[rr,"\kgfpq{p}{q}"']\&\& \Cat(\pq{\clC}[q][p],\textsf{Set})
						\end{tikzcd}
					\]
				}
			\end{adjustbox}
		\end{center}
        commute:
		\[ 	\clD(\ckgpq{p}{q}{F},1) \cong \kgpq{q}{p}{\clD(F,1)} \qquad \qquad
			\clD(1,\kgpq{p}{q}{D}) \cong \kgpq{p}{q}{\clD(1,D)}.\]
		%  \begin{align*}
		%       h_{\kgpq{p}{q}{D}}  & \cong \kgpq{p}{q}{h_{D}},  \\
		%       h^{\ckgpq{p}{q}{D}} & \cong \kgpq{q}{p}{h^{D}}.
		%   \end{align*}
		\item\label{co-limits-of-kusarigamas}\SloganFont{Limits of kusarigamas}We have functorial isomorphisms
		\[
			\pqEnd{p}{q}{\uA\in\clC}D^{\uA}_{\uA}   \cong \lim\left(\kgpq{p}{q}{D}\right),   \qquad
			\pqCoend{p}{q}{\uA\in\clC}D^{\uA}_{\uA} \cong \colim\left(\ckgpq{q}{p}{D}\right).
		\]
		\item\label{higher-arity-co-kusarigamas-from-1-1-co-kusarigamas}\SloganFont{Higher arity co/kusarigamas from $(1,1)$-co/kusarigamas}The cokusarigama
		\[\ckgpq{p}{q}{F}\colon\pq{\clC}[q][p]\longrightarrow\clD\]
		of a functor $F\colon\pq{\clC}\longrightarrow\clD$ is the left Kan extension of the $(1,1)$-cokusarigama of $\pqdiag^{*}(F)$ along $\Delta_{q,p}$:
		\[
			\ckgpq{p}{q}{F}
			=
			\Lan_{\Delta_{q,p}}\left(\ckgb{\Delta_{p,q}^{*}(F)}\right)
			\quad
			\begin{tikzcd}[row sep={6.3em,between origins}, column sep={8.1em,between origins}, ampersand replacement=\&]
				{}
				\&
				\pq{\clC}[q][p]
				\arrow[d, "\ckgpq{p}{q}{F}", dashed]
				\\
				\clC^{\op}\times\clC
				\arrow[ru, "\Delta_{q,p}"]
				\arrow[r, "\ckgb{\pqdiag^{*}(F)}"'{name=F}]
				\&
				\clD.
				\arrow[from=F,to=1-2,shorten=2.0em,Rightarrow,xshift=-0.0em,yshift=-0.25em]
			\end{tikzcd}
		\]
		Dually, the kusarigama
		\[\kgpq{q}{p}{G}\colon\pq{\clC}\longrightarrow\clD\]
		of $G\colon\pq{\clC}[q][p]\longrightarrow\clD$ is the right Kan extension of the $(1,1)$-kusarigama of $\Delta_{q,p}^{*}(G)$ along $\pqdiag$:
		\[
			\kgpq{q}{p}{G}
			=
			\Ran_{\Delta_{p,q}}\left(\kgb{\Delta_{q,p}^{*}(G)}\right)
			\quad
			\begin{tikzcd}[row sep={6.3em,between origins}, column sep={8.1em,between origins}, ampersand replacement=\&]
				{}
				\&
				\pq{\clC}
				\arrow[d, "\kgpq{q}{p}{G}", dashed]
				\\
				\clC^{\op}\times\clC
				\arrow[ru, "\pqdiag"]
				\arrow[r, "\kgb{\Delta_{q,p}^{*}(G)}"'{name=F}]
				\&
				\clD.
				\arrow[from=F,to=1-2,shorten=2.0em,Leftarrow,xshift=-0.0em,yshift=-0.25em]
			\end{tikzcd}
		\]
		\end{enumtag}
\end{proposition}
\label{sec:org2569e2e}
\subsection{Higher arity twisted arrow categories}
\begin{definition}\label{p-q-twisted-category}%
	The \emph{$(p,q)$-twisted arrow category} is the category $\pqTw{p}{q}{\clC}$ defined as the category of elements $\CatEl{\pq{\clC}[q][p]}{\ckgpq{p}{q}{1}}$ of $\ckgpq{p}{q}{1}$:
	\begin{center}
		\begin{tikzcd}[row sep={3.6em,between origins}, column sep={7.2em,between origins}, ampersand replacement=\&]
			\pqTw{p}{q}{\clC}
			\ar[d,two heads]
			\ar[r, "\Sigma_{(p,q)}",two heads]
			\&
			\pq{\clC}
			\ar[d,"\ckgpq{p}{q}{1}"]
			\\
			1
			\ar[ur,Rightarrow, shorten <=2em, shorten >=2em]
			\ar[r,"1"',hook]
			\&
			\textsf{Set}.
		\end{tikzcd}
	\end{center}%
\end{definition}
It is well-known that this comma object (as all categories of elements) admits an equivalent description as the comma object
\begin{center}
	\begin{tikzcd}[ampersand replacement=\&]
		\pqTw{p}{q}{\clC}
		\ar[d,two heads]
		\ar[r, "\Sigma_{(p,q)}",two heads]
		\&
		\pq{\clC}
		\ar[d,"\yo_{\pq{\clC}}"]
		\\
		1
		\ar[ur,Rightarrow, shorten=2.25em]
		\ar[r,"\lceil \ckgpq{p}{q}{1}\rceil"',hook]
		\&
		\PSh{\pq{\clC}},
	\end{tikzcd}
\end{center}%
where $\yo_{\pq{\clC}}$ is the Yoneda embedding and $\lceil \ckgpq{p}{q}{1}\rceil$ is the ``name'' of the functor $\ckgpq{p}{q}{1}$ picking out the object $\ckgpq{p}{q}{1}$ of $\PSh{\pq{\clC}}$. All in all, this means that the $(p,q)$-twisted arrow category of $\clC$ admits the following equivalent descriptions:
\begin{proposition}%[Unwinding \cref{p-q-twisted-category}]\label{unwinding-p-q-twisted-category}
	$\pqTw{p}{q}{\clC}$ can be equivalently characterised as:
	\begin{enumtag}{pqt}
		\item\label{PQT1}The full-subcategory of $\PSh{\clC}_{/\ckgpq{p}{q}{1}}$ spanned by representable presheaves, i.e.\ the category whose
		\begin{itemize}
			\item Objects are natural transformations of the form $\ph_{\bsA}\longrightarrow\ckgpq{p}{q}{1}$ with $\bsA\in\big(\pq{\clC}\big)_o$;
			\item Morphisms are natural transformations $\ph_{\bsf}\colon\ph_{\bsA}\longrightarrow\ph_{\bsB}$ such that the diagram
			      \[
				      \begin{tikzcd}[column sep=0.7em]
					      \ph_{\bsA} \ar[rr, "\ph_{\bsf}"] \ar[rd] && \ph_{\bsB} \ar[ld] \\
					      & \ckgpq{p}{q}{1} &
				      \end{tikzcd}
			      \]
			      commutes.%
		\end{itemize}
		\item\label{PQT2}The category whose
		\begin{itemize}
			\item Objects are triples $(\bsX,\bsY, t)$ where $(\bsX,\bsY)$ is an object of $\pq{\clC}$, and $t$ is an element of $\ckgpq{p}{q}{1}^{\bsX}_{\bsY}$;
			\item Morphisms are ``basepoint preserving'' morphisms $(\bsX,\bsY) \to (\bsX',\bsY')$.
		\end{itemize}
		\item\label{PQT4}The category whose
		\begin{itemize}
			\item Objects are collections $\left\{f_{ij}\colon A_{i}\longrightarrow B_{j}\right\}$ of morphisms of $\clD$ with $0\leq i\leq p$ and $0\leq j\leq q$;
			\item Morphisms are collections of factorisations of the codomain through the domain, of the form
			      \[
				      \begin{tikzcd}[row sep={4.5em,between origins}, column sep={4.5em,between origins}, ampersand replacement=\&]
					      A_{i}
					      \arrow[r,"f"]
					      \&
					      B_{j}
					      \arrow[d,"\psi_{j}"]
					      \\
					      A_{i}'
					      \arrow[u,"\phi_{i}"]
					      \arrow[r,"g"']
					      \&
					      B_{j}',
				      \end{tikzcd}
			      \]
			      one for each $0\leq i\leq p$ and each $0\leq j\leq q$.
		\end{itemize}
	\end{enumtag}
	If $\clC$ has finite products and coproducts, we gain an additional equivalent description of $\pqTw{p}{q}{\clC}$:
	\begin{enumtag}{twd}
		\item\label{PQT3}The category whose
		\begin{itemize}
			\item Objects are morphisms $A_{1}\coprod\cdots\coprod A_{p}\longrightarrow B_{1}\times\cdots\times B_{q}$;
			\item Morphisms are factorisations of the codomain through the domain, of the form
			      \[
				      \begin{tikzcd}[row sep={4.5em,between origins}, column sep={9.0em,between origins}, ampersand replacement=\&]
					      A_{1}\coprod\cdots\coprod A_{p}
					      \arrow[r,"f"]
					      \&
					      B_{1}\times\cdots\times B_{q}
					      \arrow[d,"\psi_{1}\times\cdots\times\psi_{q}"]
					      \\
					      \arrow[u,"\phi_{1}\coprod\cdots\coprod\phi_{p}"]
					      A_{1}'\coprod\cdots\coprod A_{p}'
					      \arrow[r,"g"']
					      \&
					      B_{1}'\times\cdots\times B_{q}'.
				      \end{tikzcd}
			      \]
		\end{itemize}
	\end{enumtag}
\end{proposition}
From this, 
\begin{align*}
	\pqEnd{p}{q}{A\in\clC}D^{\uA}_{\uA}   & \cong \lim\Big(\pqTw{p}{q}{\clC}\xloongrightarrow{\Sigma_{(p,q)}}\pq{\clC}\xloongrightarrow{D}\clD\Big),               \\
	\pqCoend{p}{q}{A\in\clC}D^{\uA}_{\uA} & \cong \colim\Big(\pqTw{p}{q}{\clC^{\op}}^{\op}\xloongrightarrow{\Sigma_{(p,q)}}\pq{\clC}\xloongrightarrow{D}\clD\Big).
\end{align*}
\section{Future work (?)}
\label{sec:org06e498e}
\subsection{weighing co/ends: the full story}
This and that
\subsection{kusarigamas are a toy example of "extradiagonal" (for lack of a better name) transformation}
This and that 
\subsection{A graphical language for higher arity co/ends.} This and that



\end{document}
